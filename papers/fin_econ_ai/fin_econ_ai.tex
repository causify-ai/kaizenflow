\documentclass[11pt, reqno]{amsart}
\usepackage{amsfonts, amssymb, amscd, amsrefs}
\usepackage{graphicx}
\usepackage{hyperref}
\usepackage{slashed}
\usepackage{fullpage}
% Prevent table repositioning.
\usepackage{float}
% For textcolor.
\usepackage{xcolor}
% For blackboard bold `1`.
\usepackage{bbold}

% TODO(gp): It aborts with
% ! Undefined control sequence.
%<argument> ...on\endcsname \protect \@secnumpunct
%
%l.164 \subsection{Breaking the symmetry}
% https://tex.stackexchange.com/questions/165930/bold-and-italic-subsection-title-with-custom-font-size
%\usepackage{titlesec}
%\titleformat{\section}
%{\normalfont\fontfamily{phv}\fontsize{12}{17}\bfseries}{\thesection}{1em}{}
%\titleformat{\subsection}
%{\normalfont\fontfamily{phv}\fontsize{12}{17}\bfseries\itshape}{\thesubsection}{1em}{}
%\titleformat{\subsection}
%  {\normalfont\fontsize{12}{17}\sffamily\bfseries\slshape}
%  {\thesubsection}
%  {1em}
%  {}

%\usepackage{titlesec}

%\titleformat*{\section}{\LARGE\bfseries}
%\titleformat*{\subsection}{\Large\bfseries}
%\titleformat*{\subsubsection}{\large\bfseries}
%\titleformat*{\paragraph}{\large\bfseries}
%\titleformat*{\subparagraph}{\large\bfseries}

% https://ctan.math.washington.edu/tex-archive/macros/latex/required/amscls/doc/amsthdoc.pdf
\newtheorem{thm}{Theorem}
\theoremstyle{definition}
\newtheorem{defn}{Definition}[subsection]
\newtheorem{problem}{Problem}[subsection]
\theoremstyle{remark}
\newtheorem{exmp}{Example}[subsection]
\newtheorem{rmk}{Remark}[subsection]

\newcommand{\bidbtc}{\mathrm{bid}_\mathrm{BTC}}
\newcommand{\askbtc}{\mathrm{ask}_\mathrm{BTC}}
\newcommand{\bideth}{\mathrm{bid}_\mathrm{ETH}}
\newcommand{\asketh}{\mathrm{ask}_\mathrm{ETH}}

\newcommand{\bidbase}{\mathrm{bid}_\mathrm{quote\;token}}
\newcommand{\askbase}{\mathrm{ask}_\mathrm{quote\;token}}
\newcommand{\bidquote}{\mathrm{bid}_\mathrm{base\;token}}
\newcommand{\askquote}{\mathrm{ask}_\mathrm{base\;token}}
% Use teletype for tokens (\texttt{...}), but do not allow italics
%  (\textnormal{...}).
\newcommand{\BTC}{\textnormal{\texttt{wBTC}}}
\newcommand{\ETH}{\textnormal{\texttt{ETH}}}
\newcommand{\DAI}{\textnormal{\texttt{DAI}}}
\newcommand{\USDC}{\textnormal{\texttt{USDC}}}
\newcommand{\USDT}{\textnormal{\texttt{USDT}}}
\newcommand{\tA}{\textnormal{\texttt{A}}}
\newcommand{\tB}{\textnormal{\texttt{B}}}
% Use non-italic teletype for order attributes in order notation.
\newcommand{\timestamp}{\textnormal{\texttt{timestamp}}}
\newcommand{\action}{\textnormal{\texttt{action}}}
\newcommand{\quantity}{\textnormal{\texttt{quantity}}}
\newcommand{\basetoken}{\textnormal{\texttt{base\_token}}}
\newcommand{\limitprice}{\textnormal{\texttt{limit\_price}}}
\newcommand{\quotetoken}{\textnormal{\texttt{quote\_token}}}
\newcommand{\depositaddress}{\textnormal{\texttt{deposit\_address}}}
\newcommand{\zkpcomplexity}{\textnormal{\texttt{zkp\_complexity}}}
\newcommand{\zkpdeadline}{\textnormal{\texttt{zkp\_deadline}}}
%
\newcommand{\buy}{\textnormal{\texttt{buy}}}
\newcommand{\sell}{\textnormal{\texttt{sell}}}
\newcommand{\nan}{\textnormal{\texttt{nan}}}
%
\newcommand{\midpoint}{\mathrm{midpoint}}
% Table icons.
\newcommand{\trm}{\textcolor{red}{-}}
\newcommand{\tyo}{\textcolor{yellow}{o}}
\newcommand{\tgp}{\textcolor{green}{+}}

% Use this if using `\contrib[]{...}`.
%\makeatletter\let\@wraptoccontribs\wraptoccontribs\makeatother
% https://tex.stackexchange.com/questions/418547/equal-contribution-using-thanks-with-llncs-class#418563
\makeatletter
\newcommand{\printfnsymbol}[1]{%
\textsuperscript{\@fnsymbol{#1}}%
}
\makeatother

\begin{document}
	\title{A Path Towards Artificial General Intelligence in Finance}

	\author{Giacinto Paolo Saggese$^{*}$}
	\author{Paul Smith$^{*}$}
	\thanks{$^{*}$ Authors listed alphabetically.}

	\date{\today}

	\begin{abstract}
		This is draft v0.1.
	\end{abstract}

	\maketitle

	\setcounter{tocdepth}{2}
	\tableofcontents

	% ###############################################################################
	\section{Introduction}

    This paper proposes an approach to forecast financial and economic
    variables combining several emerging technologies:
		\begin{enumerate}
			\item Deep-learning based AI, in particular large language models (LLMs)

			\item Knowledge graphs (KGs)

			\item Bayesian inference for probabilistic reasoning

			\item High-performance streaming computing
		\end{enumerate}

    The proposed approach allows to deal with most of the problems plaguing
    financial and economic forecasting such as:
    \begin{itemize}
      \item Small and noisy datasets
      \item Need for interpretability of forecasts
      \item Need for estimation of confidence in prediction
      \item Complex, time-varying relationships between variables
    \end{itemize}

    It also allows to deal with problems of:
    \begin{itemize}
      \item LLM hallucinations
      \item Lack of fit (overfit or underfit) in machine learning models
    \end{itemize}

	% ###############################################################################
	\section{A possible definition of AGI for financial applications}

	We define AGI as a computer system able to perform tasks better than a
	human can do.

	Out approach to developing AGI, with an initial focus on finance,
	reflects a strategic pathway towards achieving broader artificial general
	intelligence capabilities. Here's an analysis of your approach and its implications:

	\begin{itemize}
		\item \textbf{Domain-Specific Expertise as a Foundation for AGI}: Focusing
			on finance initially leverages the concept that substantial domain-specific
			knowledge is essential for creating super-human performance. Finance,
			with its complexity and data-rich environment, provides a fertile
			ground for developing and refining AI algorithms. This sector demands
			precision, adaptability, and the ability to process vast amounts of data
			– attributes that are foundational for AGI.

		\item \textbf{Stepping Stone Approach}: By mastering one domain, you
			establish a methodology and technological base that can be adapted
			and expanded to other domains. This step-by-step approach is pragmatic,
			reducing the initial complexity that would come with attempting to
			create a multi-domain AGI from the start. Each domain mastered adds another
			layer of complexity and capability to the AGI, gradually moving
			towards true general intelligence.

		\item \textbf{Superhuman Performance in Finance}: An AGI with superhuman
			capabilities in finance would handle tasks such as predictive analytics,
			risk assessment, market trend analysis, and financial forecasting
			with unprecedented accuracy and speed. This would not only revolutionize
			the financial industry but also serve as a proof of concept for AGI's
			potential in other areas.

		\item \textbf{Integration of Diverse Data Sources and Techniques}: The ability
			to integrate and analyze data from various sources using advanced modeling
			techniques is crucial. This includes real-time processing of global
			market data, news, economic indicators, and even sociopolitical events.
			Such integrative capacity is essential for AGI, as it mirrors the multifaceted
			nature of human intelligence.

		\item \textbf{Addressing the Certainty and Hallucination Problem}: In AI,
			hallucination refers to generating or inferring information that isn't
			supported by data. An essential feature of your AGI system in finance
			would be its ability to estimate and report the level of certainty in
			its predictions and decisions. This is a significant challenge in AI
			development, as it requires the system to not only analyze data but also
			understand and communicate the limits of its knowledge and the confidence
			in its conclusions.

		\item \textbf{Ethical and Practical Implications}: As your AGI system advances,
			it's important to consider ethical implications, particularly in sensitive
			areas like finance. Issues such as data privacy, security, and the
			potential impact on employment and economic systems are vital
			considerations. Additionally, the system's decisions and predictions
			should be transparent and explainable, especially when they significantly
			impact financial markets or individual investments.
	\end{itemize}

	Questions should be answered estimating a measure of certainty (this is related
	to the problem of hallucination)

	The goal is a system that can automatically answer questions with super-human
	accuracy like:
	\begin{itemize}
		\item Can you analyze the recent performance of [specific stock or
			sector]?

		\item How do interest rate changes affect the bond market right now?

		\item What will the rate of inflation in the next year in Italy?
            The current answer from ChatGPT is:
            ```Predicting the exact rate of inflation for the next year in any
            country, including Italy, is a complex task that involves analyzing
            a multitude of economic indicators, current monetary policies,
            global economic trends, and unforeseen events. As of my last update
            in April 2023, I don't have real-time data or the ability to
            predict future economic conditions.```

		\item Can you predict the risk of a specific stock, such as Apple, in the
			next 15 minutes based on real-time market data, recent company news, and
			advanced predictive analytics? \\
            The current answer from ChatGPT is: ```You
			are an expert economist. Can you predict the risk of a specific stock,
			such as Apple, in the next 15 minutes based on real-time market data,
			recent company news, and advanced predictive analytics? Answer in 30 words```
			ChatGPT: As an AI, I don't have real-time market data or the ability to
			predict short-term stock movements. Accurate 15-minute forecasts
			require live data and are highly speculative due to market volatility.

		\item What would be the global economic impact of a sudden 2\% increase
			in the U.S. Federal Reserve interest rate today?

		\item What portfolio composition would maximize returns for a given risk
			level over the next year, considering global economic forecasts and
			market volatility predictions? ChatGPT: Optimal portfolio composition
			varies; it depends on risk tolerance, market trends, and diversification
			across asset classes, considering economic forecasts and volatility
			projections.

		\item Which industries are most likely to see significant mergers and acquisitions
			in the next year, and what will be the potential impact on the market?

		\item What are the predicted movements of major currencies in the forex
			market over the next quarter, based on current geopolitical
			situations, trade relations, and economic policies? ChatGPT:
			Predicting forex market movements involves analyzing geopolitical
			tensions, trade agreements, central bank policies, and economic indicators,
			but it's inherently uncertain due to the market's volatility and
			unforeseen events.
	\end{itemize}

	% ================================================================================
	\subsection{The need for incorporating certainty}

	Implementing a measure of certainty in AGI responses, particularly in a complex
	field like finance, is therefore not just a technical feature, but a fundamental
	aspect that enhances the utility, reliability, and ethical use of the system.

	Incorporating a measure of certainty in the responses of an AGI,
	especially in the context of finance, is crucial for several reasons:

	\begin{itemize}
		\item \textbf{Reducing the Risk of Hallucination}: AI systems,
			including AGIs, can sometimes "hallucinate", meaning they generate
			responses or predictions that are not grounded in the data they have
			processed. By quantifying the certainty of its answers, an AGI can
			indicate the reliability of its predictions, helping users differentiate
			between high-confidence insights and those that are more speculative.

		\item \textbf{Enhancing Decision-Making}: In finance, decisions often
			hinge on the level of risk and uncertainty. An AGI that can estimate and
			communicate the degree of certainty in its analysis provides invaluable
			information for risk assessment. This allows users to make more
			informed decisions, weighing the potential risks and rewards more accurately.

		\item \textbf{Building Trust}: A system that acknowledges the limits of
			its knowledge and provides certainty estimates can build greater trust
			with its users. In fields like finance, where decisions can have significant
			consequences, trust in the system's output is essential.

		\item \textbf{Dynamic Learning and Improvement}: By quantifying certainty,
			the AGI can also identify areas where its models may need improvement.
			Lower certainty in certain types of predictions can signal the need for
			additional data, refinement of models, or reevaluation of the algorithms
			used.

		\item \textbf{Managing Complex Systems}: Finance is a complex, dynamic
			system influenced by a multitude of factors. Providing a measure of certainty
			helps in understanding the impact of various elements and their interplay,
			acknowledging that some aspects of the financial world are inherently
			unpredictable.

		\item \textbf{Ethical and Responsible AI}: This approach aligns with ethical
			AI practices, where transparency and accountability are key. Users should
			be aware of both the capabilities and limitations of the AGI, and a
			measure of certainty facilitates this understanding.
	\end{itemize}

	% ================================================================================
	\subsection{Why focusing on finance?}

	The application of AGI in finance and economics is not only a test of its
	predictive capabilities but also an exploration into understanding complex,
	human-centric systems. The success in these fields can have far-reaching implications,
	both in terms of technological advancement and societal benefits.

	Building Artificial General Intelligence (AGI) in the field of finance
	and economics is meaningful for several reasons:

	\begin{itemize}
		\item \textbf{Understanding Complex Systems}: Finance and economics are
			intricate fields characterized by non-linear, interdependent
			variables. AGI systems could offer advanced understanding of these complexities,
			allowing for better predictions and insights. This is particularly valuable
			as financial systems are influenced by a myriad of factors including human
			behavior, market trends, political climates, and global events.

		\item \textbf{Dynamic Prediction Capabilities}: AGI in finance isn't just
			about static analysis but involves dynamic prediction in real-time.
			Financial markets are constantly evolving, and an AGI system can adapt
			to these changes, offering predictions and analyses that reflect
			current realities. This dynamic environment is a robust testing ground
			for AGI capabilities.

		\item \textbf{Managing Uncertainty and Noise}: Financial data is often noisy
			and uncertain. An AGI system's ability to sift through this noise and
			make accurate predictions despite uncertainty is a testament to its
			understanding of complex, real-world environments. This capability
			could significantly reduce risks and improve decision-making
			processes.

		\item \textbf{Immediate Practical Benefits}: Enhancements in economic and
			financial decision-making directly translate to benefits for the human
			race. Better financial predictions and economic models can lead to
			more stable economies, improved allocation of resources, reduced
			risks of financial crises, and overall economic growth. This can
			improve living standards and contribute to societal well-being.

		\item \textbf{Understanding Human Nature}: Since finance and economics are
			deeply intertwined with human behavior, an AGI capable of navigating
			these fields would necessarily gain insights into human psychology
			and behavior. This is crucial for building AGI that truly understands
			and interacts effectively with human-centric environments.

		\item \textbf{Ethical and Societal Implications}: Working in these domains
			also necessitates dealing with ethical considerations, such as
			privacy, security, and fairness. Successfully navigating these issues
			in the realm of finance and economics can set precedents for AGI applications
			in other areas, leading to more responsible and ethical AI
			development.
	\end{itemize}

	% ================================================================================
	\subsection{The problem with applying machine learning in economics and
	finance}

	Applying machine learning in economics and finance presents several
	challenges due to the unique nature of these fields, besides the traditional
	problems of overfitting (i.e., fitting the noise in the training data
	instead of capturing the underlying true dynamics, leading to poor generalization)
	and underfitting (i.e., oversimplified models may fail to capture the complexities
	of financial data, resulting in inaccurate predictions)

	Some specific reasons that make machine learning difficult to apply in
	economics and finance are:

	\begin{enumerate}
		\item \textbf{Data Quality and Availability}:
			\begin{itemize}
				\item \textbf{Noise in Data}: Economic and financial data are often
					noisy and non-stationary. Market sentiments, geopolitical events,
					and economic policies can rapidly change, affecting the quality and
					relevance of data.

				\item \textbf{Limited Access}: High-quality, granular data can be
					expensive or restricted, limiting the scope of analysis.
					Proprietary data sources and privacy concerns add to this
					challenge.
			\end{itemize}

		\item \textbf{Complexity of Financial Markets}:
			\begin{itemize}
				\item \textbf{Non-Linear Relationships}: Financial markets are influenced
					by a complex web of interrelated factors. The relationships
					between these factors are often non-linear and can change over
					time.

				\item \textbf{Market Efficiency}: Efficient Market Hypothesis suggests
					that current asset prices reflect all available information.
					Predicting future movements based on historical data can be
					challenging as past patterns might not predict future movements.

				\item \textbf{Regime Shifts}: Economic conditions and policies can change,
					leading to regime shifts. Models trained on data from one economic
					regime might perform poorly in another.

				\item \textbf{Black Swan Events}: Unpredictable events (like the
					2008 financial crisis or the COVID-19 pandemic) can dramatically shift
					economic trends, rendering existing models ineffective.

				\item \textbf{Behavioral Factors}: Economic and financial decisions
					are often influenced by human behavior, which can be irrational
					and hard to predict. Modeling these behaviors accurately is a significant
					challenge.

				\item \textbf{Non-stationarity}: non-stationarity refers to data whose
					statistical properties, (e.g., mean and variance), change over time,
					making it challenging to model and predict due to evolving trends,
					cycles, and unexpected events like market crashes or economic
					booms. \textbf{Non-Gaussianity}: non-Gaussianity of random variable
					distributions reflects irregular, often extreme events, like market
					crashes, which are not well-described by the normal distribution's
					bell curve, leading to heavier tails and more pronounced risks than
					Gaussian models would suggest.
			\end{itemize}

		\item \textbf{Regulatory and Ethical Considerations}:
			\begin{itemize}
				\item Financial models are subject to regulatory scrutiny. Machine
					learning models, often seen as `black boxes', can raise concerns
					about transparency, accountability, and fairness.

				\item Ethical concerns, such as the risk of amplifying biases in
					financial decision-making, are also significant.
			\end{itemize}

		\item \textbf{Time Series Analysis Challenges}: Economic and financial data
			are typically time-series, which have their own challenges like
			autocorrelation, trend/cycle extraction, and handling of non-stationary
			data.

		\item \textbf{Generalization and Scalability}: Models trained on data from
			specific markets or periods may not generalize well across different contexts
			or times.

		\item \textbf{Feedback Loops}: Actions based on model predictions can influence
			the market, creating a feedback loop that can invalidate the model's
			assumptions.

		\item \textbf{Model Interpretability}: There's often a trade-off between
			model complexity and interpretability. In finance and economics, understanding
			the `why' behind a prediction can be as important as the prediction itself.
	\end{enumerate}

	% ###############################################################################
    \section{Description of the architecture}

    - Offline pass
      - Step 1: generate a domain-specific knowledge graphs
        - We investigate automatic ways (e.g., using LLMs to generate a knowledge
          graphs of the domain)
        - The knowledge graph contains dependencies:
          - Causal or correlation
          - Formula representing the type of relationship (sign, linear)
          - Strength of the relationship (in terms of belief)
        - Human to "regularize", inspect it, improve it

    - Online pass
      - Step 1: user poses a question to system
      - Step 2: an LLM answers a question by extracting a subset of the KG
      - Step 3: the subset of KG is converted into a Bayesian network
      - Step 4: Bayesian model is built and evaluated
        - Timeseries data is accessed
        - E.g., we use KaizenFlow framework to build, test, and deploy
      - Step 5: the answer to the question is returned together with estimates
        about the error and the epistemological conviction

	% ###############################################################################
	\section{Large language models}

	% ###############################################################################
	\section{Knowledge graphs}

    Knowledge graphs (KGs) store structured knowledge as a collection of triples
    $$KG = \{(h, r, t) \subset \Epsilon \times R \times E\}$$
    where
    - $E$ is a set of entities
    - $R$ is a set of relations

    Domain-specific KGs are constructued to represent knowledge in a specific
    domain (e.g., medical, biology, and finance)

    % TODO(gp): Describe

	% ###############################################################################
	\section{Merging large Language models and knowledge graphs}

    \subsection{Pros and cons of LLMs}

    Large language models, pre-trained on large-scale corpora, have shown great
    performance in many natural language processing (NLP) tasks. By increasing
    model size, LLMs with billions of parameters (e.g., ChatGPT and PaLM2) have
    shown a surprising emergent ability and generalizability.

    Despite their success in many applications, LLMs have been criticized for
    several limitations.

    Advantages of LLMs
    - General knowledge
    - Language processing
    - Generalizability

    Limitations of LLMs
    - Black-box: lack of interpretability
    - Implicit knowledge: LLMs are black-box models, which fall short of
      capturing and accessing factual knowledge
    - Hallucination
    - Indecisiveness (reasoning happens through probabilistic process)
    - Lack of domain-specific knowledge
    - Lack of new knowledge

    \subsection{Pros and cons of KGs}

    A potential solution is to incorporate knowledge graphs (KGs) into LLMs.
    
    Advantages of KGs
    - Structural knowledge
    - Accuracy
    - Decisiveness
    - Interpretability
    - Domain-specific knowledge
    - Evolving knowledge

    Limitations of KGs
    - Incomplete knowledge
    - Lack language understanding
    - Unseen facts

    LLMs and KGs are two inherently complementary techniques, which should be
    unified into a general framework to mutually enhance each other.

	% ###############################################################################
	\section{Bayesian networks}

    % TODO(gp): Describe

	% ###############################################################################
	\section{Building Bayesian model}

    % TODO(gp): Describe

	% ###############################################################################
	\section{Running the model}

    % TODO(gp): Describe

	% ###############################################################################
	\bibliography{Bib}
	\bibliographystyle{amsplain}

	\begin{thebibliography}{10}
		\bib{TrDeProv23}{article}{ author={Trace}, title={Decentralized Proving, Proof Markets, and ZK Infrastructure}, date={June 2023}, eprint={https://figmentcapital.medium.com/decentralized-proving-proof-markets-and-zk-infrastructure-f4cce2c58596} }

		%    \bib{}{article}{
		%      author={,},
		%      title={},
		%      date={},
		%    }
	\end{thebibliography}
\end{document}
