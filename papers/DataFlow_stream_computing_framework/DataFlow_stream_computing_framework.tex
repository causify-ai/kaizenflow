\documentclass[11pt, reqno]{amsart}
\usepackage{amsfonts, amssymb, amscd, amsrefs}
\usepackage{graphicx}
\usepackage{hyperref}
\usepackage{slashed}
\usepackage{fullpage}
% Prevent table repositioning.
\usepackage{float}
% For textcolor.
\usepackage{xcolor}
% For blackboard bold `1`.
\usepackage{bbold}
\usepackage{tikz}
\usepackage[pdf]{graphviz}
%
% https://tex.stackexchange.com/questions/83882/how-to-highlight-python-syntax-in-latex-listings-lstinputlistings-command
% https://www.overleaf.com/learn/latex/Code_listing
\usepackage{listings}
% Default fixed font does not support boldface
\DeclareFixedFont{\ttb}{T1}{txtt}{bx}{n}{12} % for bold
\DeclareFixedFont{\ttm}{T1}{txtt}{m}{n}{12}  % for normal

% Custom colors
\usepackage{color}
\definecolor{deepblue}{rgb}{0,0,0.5}
\definecolor{deepred}{rgb}{0.6,0,0}
\definecolor{deepgreen}{rgb}{0,0.5,0}

\usepackage{listings}

% Python style for highlighting
\newcommand\pythonstyle{\lstset{
  language=Python,
  basicstyle=\ttm,
  morekeywords={self},              % Add keywords here
  keywordstyle=\ttb\color{deepblue},
  emph={MyClass,__init__},          % Custom highlighting
  emphstyle=\ttb\color{deepred},    % Custom highlighting style
  stringstyle=\color{deepgreen},
  frame=tb,                         % Any extra options here
  showstringspaces=false
}}

% Python environment.
\lstnewenvironment{python}[1][]
{
\pythonstyle
\lstset{#1}
}
{}

% Python for external files.
\newcommand\pythonexternal[2][]{{
\pythonstyle
\lstinputlisting[#1]{#2}}}

% Python for inline.
\newcommand\pythoninline[1]{{\pythonstyle\lstinline!#1!}}


\usepackage{listings}
\usepackage{xcolor}

\definecolor{codegreen}{rgb}{0,0.6,0}
\definecolor{codegray}{rgb}{0.5,0.5,0.5}
\definecolor{codepurple}{rgb}{0.58,0,0.82}
\definecolor{backcolour}{rgb}{0.95,0.95,0.92}

\lstdefinestyle{mystyle}{
    backgroundcolor=\color{backcolour},
    commentstyle=\color{codegreen},
    keywordstyle=\color{magenta},
    numberstyle=\tiny\color{codegray},
    stringstyle=\color{codepurple},
    basicstyle=\ttfamily\footnotesize,
    breakatwhitespace=false,
    breaklines=true,
    captionpos=b,
    keepspaces=true,
    numbers=left,
    numbersep=5pt,
    showspaces=false,
    showstringspaces=false,
    showtabs=false,
    tabsize=2
}

\lstset{style=mystyle}

% end python

% TODO(GP): It aborts with
% ! Undefined control sequence.
%<argument> ...on\endcsname \protect \@secnumpunct
%
%l.164 \subsection{Breaking the symmetry}
% https://tex.stackexchange.com/questions/165930/bold-and-italic-subsection-title-with-custom-font-size
%\usepackage{titlesec}
%\titleformat{\section}
%{\normalfont\fontfamily{phv}\fontsize{12}{17}\bfseries}{\thesection}{1em}{}
%\titleformat{\subsection}
%{\normalfont\fontfamily{phv}\fontsize{12}{17}\bfseries\itshape}{\thesubsection}{1em}{}
%\titleformat{\subsection}
%  {\normalfont\fontsize{12}{17}\sffamily\bfseries\slshape}
%  {\thesubsection}
%  {1em}
%  {}

%\usepackage{titlesec}

%\titleformat*{\section}{\LARGE\bfseries}
%\titleformat*{\subsection}{\Large\bfseries}
%\titleformat*{\subsubsection}{\large\bfseries}
%\titleformat*{\paragraph}{\large\bfseries}
%\titleformat*{\subparagraph}{\large\bfseries}

% https://ctan.math.washington.edu/tex-archive/macros/latex/required/amscls/doc/amsthdoc.pdf
\newtheorem{thm}{Theorem}
\theoremstyle{definition}
\newtheorem{defn}{Definition}[subsection]
\newtheorem{problem}{Problem}[subsection]
\theoremstyle{remark}
\newtheorem{exmp}{Example}[subsection]
\newtheorem{rmk}{Remark}[subsection]

% Use teletype for tokens (\texttt{...}), but do not allow italics
%  (\textnormal{...}).
\newcommand{\USDT}{\textnormal{\texttt{USDT}}}
\newcommand{\tA}{\textnormal{\texttt{A}}}
\newcommand{\tB}{\textnormal{\texttt{B}}}
%
\newcommand{\nan}{\textnormal{\texttt{nan}}}
\newcommand{\NaN}{\textnormal{\texttt{NaN}}}
% Dataframe.
\newcommand{\df}{\mathbf{df}}
%
% Table icons.
\newcommand{\trm}{\textcolor{red}{-}}
\newcommand{\tyo}{\textcolor{yellow}{o}}
\newcommand{\tgp}{\textcolor{green}{+}}

% Use this if using `\contrib[]{...}`.
%\makeatletter\let\@wraptoccontribs\wraptoccontribs\makeatother
% https://tex.stackexchange.com/questions/418547/equal-contribution-using-thanks-with-llncs-class#418563
\makeatletter
\newcommand{\printfnsymbol}[1]{%
  \textsuperscript{\@fnsymbol{#1}}%
}
\makeatother

\begin{document}

\title{KaizenFlow: a framework for high-performance machine learning stream computing}

\author{Giacinto Paolo Saggese$^{*}$}
\author{Paul Smith$^{*}$}
\thanks{$^{*}$ Authors listed alphabetically.}

\thanks{With contributions from
  Shayan Ghasemnezhad,
  Danil Iachmenev,
  Tamara Jordania,
  Sonaal Kant,
  Samarth KaPatel,
  Grigorii Pomazkin,
  Sameep Pote,
  Juraj Smeriga,
  Daniil Tikhomirov,
  Nina Trubacheva,
  and
  Vladimir Yakovenko,
}

\date{\today}

\begin{abstract}
  This is draft v0.2.
\end{abstract}

\maketitle

\setcounter{tocdepth}{2}
\tableofcontents

% ###############################################################################
\section{Introduction}

KaizenFlow is a framework to build, test, and deploy high-performance streaming
computing systems based on machine learning and artificial intelligence.

The goal of KaizenFlow is to increase the productivity of data scientists by
empowering them to design and deploy systems with minimal or no
intervention from data engineers and devops support.

Guiding desiderata in the design of KaizenFlow include:
\begin{enumerate}
  \item Support rapid and flexible prototyping with the standard
        Python/Jupyter/data science tools
  \item Process both batch and streaming data in exactly the same way
  \item Avoid software rewrites in going from prototype to production
  \item Make it easy to replay stream events in testing and debugging
  \item Specify system parameters through config
  \item Scale gracefully to large data sets and dense compute
\end{enumerate}

These design principles are embodied in the many design features of KaizenFlow,
which include:
\begin{enumerate}
  \item \textbf{Computation as a direct acyclic graph}.
        KaizenFlow represents models as direct acyclic graphs (DAG), which is
        a natural form for dataflow and reactive models typically found in
        real-time systems. Procedural statements are also allowed inside
        nodes.
  \item \textbf{Time series processing}.
        All KaizenFlow components (such as data store, compute engine,
        deployment) handle time series processing in a native way.
        Each time series can be univariate or multivariate (e.g., panel data)
        represented in a data frame format.
  \item \textbf{Support for both batch and streaming modes}.
        The framework allows running a model both in batch and streaming mode,
        without any change in the model representation.
        The same compute graph can be executed feeding data in one shot or in
        chunks (as in historical/batch mode), or as data is generated (as in
        streaming mode).
        KaizenFlow guarantees that the model execution is the same
        independently on how data is fed, as long as the model is strictly
        causal. Testing frameworks are provided to compare batch/streaming
        results so that any causality issues may be detected early in the
        development process.
  \item \textbf{Precise handling of time}.
        All components automatically track the knowledge time of when the data
        is available at both their input and output. This allows one to easily
        catch future peeking bugs, where a system is non-causal and uses data
        available in the future.
  \item \textbf{Observability and debuggability}.
        Because of the ability to capture and replay the execution of any
        subset of nodes, it is possible to easily observe and debug the
        behavior of a complex system.
  \item \textbf{Incremental computation and caching}.
        Because the dependencies between nodes are explicitly tracked by
        KaizenFlow, only nodes that see a change of inputs or in the
        implementation code need to be recomputed, while the redundant
        computation can be automatically cached.
  \item \textbf{Maximum parallelism}.
        Because the computation is expressed as a DAG, the KaizenFlow
        execution scheduler can extract the maximum amount of parallelism and
        execute multiple nodes in parallel in a distributed fashion,
        minimizing latency and maximizing throughput of a computation.
  \item \textbf{Configured through a hierarchical configuration}.
        Each parameter in a KaizenFlow system is controlled by a
        corresponding value in a configuration.
        In other words, the config space is homeomorphic with the space of
        KaizenFlow systems: each config corresponds to a unique KaizenFlow
        system, and vice versa, each KaizenFlow sytem is completely
        represented by a Config.
        A configuration is represented as a nested dictionary following the
        same structure of the DAG to make it easy to navigate its structure
        This makes it easy to create an ensemble of DAGs sweeping a multitude
        of parameters to explore the design space.
  \item \textbf{Tiling}.
        KaizenFlow's framework allows streaming data with different tiling
        styles (e.g., across times, across features, and both), to minimize
        the amount of working memory needed for a given computation, increasing
        the chances of caching computation.
  \item \textbf{Support for train/prediction mode}.
        A DAG can be run in `fit' mode to learn some parameters, which are
        stored by the relevant DAG nodes, and then run in `predict` mode
        to use the learned parameters to make predictions.
        This mimics the Sklearn semantic.
        There is no limitation to the number of evaluation phases that can be
        created (e.g., train, validation, prediction, save state, load state).
        Many different learning styles are supported from different types of
        runners (e.g., in-sample-only, in-sample vs out-of-sample,
        rolling learning, cross-validation).
  \item \textbf{Model serialization}.
        A fit DAG can be serialized to disk and then materialized for
        prediction in production.
  \item \textbf{Automatic vectorization}.
        KaizenFlow DAG nodes can apply a computation to a cross-section of
        features relying on numpy and Pandas vectorization.
  \item \textbf{Deployment and monitoring}.
        A KaizenFlow system can be deployed as a Docker container.
        Even the development system is run as a Docker container, supporting
        the development and testing of systems on both cloud (e.g., AWS) and
        local desktop. Airflow is used to schedule and monitor long-running
        KaizenFlow systems.
  \item \textbf{Python and Jupyter friendly}.
        The framework is completely implemented in high-performance Python.
        It supports natively `asyncio' to overlap computation and I/O. The
        same DAG can be run in a Jupyter notebook for research and
        experimentation or in a production script, without any change in
        code.
  \item \textbf{Python data science stack support}.
        Data science libraries (such as Pandas, numpy, scipy, sklearn) are
        supported natively to describe computation. The framework comes with
        a library of pre-built nodes for many ML/AI applications.
\end{enumerate}

% ================================================================================
\subsection{The fallacy of finite data in data science}

Data scientists read a set of files as input, build a model, and produce a new
set of output files from the input files and the model.
The big assumption by data scientists is that the input is bounded, i.e., of a
known and finite size. In reality most of data is unbounded because it arrives
gradually over time.
Users produce data yesterday and today, and will produce more data tomorrow.
The data is never complete.

Data scientists deal with this problem splitting artificially the data in chunks
and then applying their modeling and transformactions to these data chunks.

The problem with this approach:
- micro-batching
- future peeking

Rewrite model.

In this case the model

% ================================================================================
\subsection{Definition of stream computing}

In the computer science literature, several terms (such as event/data stream
processing, graph computing, dataflow computing, reactive computing) are used
to describe what in this paper we refer succintly to as ``stream computing".

By stream processing we refer to a programming paradigm where streams of data
(e.g., time series or dataframes) are the objects of computation.
A series of operations (aka kernel functions) are applied to each element in
the stream.

Stream computing represents a paradigm shift from traditional batch processing
and imperative languages, emphasizing real-time data handling, adaptability,
and parallel processing, making it highly effective for modern data-intensive
applications.

% ================================================================================
\subsection{Core principles of stream computing}

The core principles of stream computing are:

\begin{enumerate}
  \item \textbf{Node-based architecture}.
        In stream and dataflow programming, the code is structured as a network of
        nodes. Each node represents a computational operation or a data processing
        function. Nodes are connected by edges that represent data streams.

  \item \textbf{Data-driven and reactive execution}.
        Execution in dataflow languages is data-driven, meaning that a node will
        process data as soon as it becomes available. Unlike imperative languages
        where the order of operations is predefined, in dataflow languages, the
        flow of data determines the order of execution.

  \item \textbf{Automatic parallelism}.
        The Dataflow programming paradigm naturally lends itself to parallel
        execution. Since nodes operate independently, they can process different
        data elements simultaneously, exploiting concurrent processing
        capabilities of modern hardware.

  \item \textbf{Continuous data streams}.
        Streams represent a continuous flow of data rather than discrete batches. Nodes
        in the network continuously receive, process, and output data, making them
        ideal for real-time data processing.

  \item \textbf{State management}.
        Nodes can be stateful or stateless. Stateful nodes retain information about
        previously processed data, enabling complex operations like windowing,
        aggregation, or pattern detection over time.

  \item \textbf{Compute intensity}.
        Stream processing is characterized by a high number of arithmetic
        operations per I/O and memory reference (e.g., it can be 50:1), since
        the same kernel is applied to all records and a number of records can be
        processed simultaneously. Furthermore, data is produced once and read only
        a few times.

  \item \textbf{Dynamic adaptability}.
        The dataflow model can dynamically adapt to changes in the data stream
        (like fluctuations in volume or velocity), ensuring efficient processing
        under varying conditions.

  \item \textbf{Scalability}.
        The model scales well horizontally, meaning one can add more nodes (or
        resources) to handle increased data loads without major architectural
        changes.

  \item \textbf{Event-driven processing}.
        Many dataflow languages support event-driven models where specific events
        in the data stream can trigger particular computational pathways or nodes.
\end{enumerate}

% ================================================================================
\subsection{Applications of stream computing}

Stream computing is a natural solution in a wide range of industries and
scenarios, as described below.

\begin{enumerate}
  \item Generic machine learning:
        \begin{itemize}
          \item Graph-based computations are ubiquitous in modern deep learning.
        \end{itemize}

  \item Financial Services
        \begin{itemize}
          \item Trading: Analyzing market data in real-time to make automated trading
                decisions.
          \item Fraud Detection: Monitoring transactions as they happen to detect and
                prevent fraudulent activities.
          \item Risk Management: Real-time assessment of financial risks based on
                current market conditions and ongoing transactions.
        \end{itemize}

  \item Internet of Things (IoT):
        \begin{itemize}
          \item Smart Homes: Processing data from various home devices for automation
                and monitoring.
          \item Industrial IoT: Real-time monitoring and control of industrial
                equipment and processes.
          \item Smart Cities: Integrating data from traffic, public services, and
                environmental sensors to optimize urban management.
        \end{itemize}

  \item Telecommunications:
        \begin{itemize}
          \item Network Monitoring and Optimization: Analyzing traffic patterns to
                optimize network performance and detect anomalies.
          \item Customer Experience Management: Real-time analysis of customer data to
                improve service and personalize offerings.
        \end{itemize}

  \item Healthcare:
        \begin{itemize}
          \item Remote Patient Monitoring: Continuous monitoring of patient vitals for
                timely medical intervention.
          \item Real-Time Health Data Analysis: Analyzing data streams from medical
                devices for immediate clinical insights.
        \end{itemize}

  \item Retail and E-Commerce:
        \begin{itemize}
          \item Personalized Recommendations: Real-time analysis of customer behavior
                to offer personalized product recommendations.
          \item Supply Chain Optimization: Streamlining logistics and inventory
                management based on real-time data.
        \end{itemize}

  \item Media and Entertainment:
        \begin{itemize}
          \item Content Optimization: Real-time analysis of viewer preferences and
                behavior for content recommendations.
          \item Live Event Analytics: Monitoring and analyzing data from live events
                for audience engagement and operational efficiency.
        \end{itemize}

  \item Transportation and Logistics:
        \begin{itemize}
          \item Fleet Management: Tracking and managing vehicles in real time for
                optimal routing and scheduling.
          \item Predictive Maintenance: Analyzing data from transportation systems to
                predict and prevent equipment failures.
        \end{itemize}

  \item Energy and Utilities:
        \begin{itemize}
          \item Smart Grid Management: Balancing supply and demand in real-time and
                identifying grid anomalies.
          \item Renewable Energy Optimization: Optimizing the output of renewable
                energy sources by analyzing environmental data streams.
        \end{itemize}

  \item Cybersecurity:
        \begin{itemize}
          \item Intrusion Detection Systems: Real-time monitoring of network traffic
                to detect and respond to cyber threats.
          \item Threat Intelligence: Analyzing global cyber threat data streams for
                proactive security measures.
        \end{itemize}

  \item Environmental Monitoring:
        \begin{itemize}
          \item Climate and Weather Analysis: Processing data from environmental
                sensors for weather prediction and climate research.
          \item Pollution Monitoring: Real-time tracking of air and water quality.
        \end{itemize}

  \item Gaming:
        \begin{itemize}
          \item In-Game Analytics: Real-time analysis of player behavior for game
                optimization and personalized experiences.
        \end{itemize}

  \item Social Media Analytics:
        \begin{itemize}
          \item Trend Analysis: Monitoring social media streams to identify and
                analyze trending topics and sentiments.
        \end{itemize}

  \item Emergency Response:
        \begin{itemize}
          \item Disaster Monitoring and Management: Real-time data analysis for
                effective response during natural or man-made disasters.
        \end{itemize}
\end{enumerate}

% ###############################################################################
\section{Configuration layer}

% --------------------------------------------------------------------------------
\subsubsection{Desirable properties of a configuration layer}
Configuring complex systems is not a trivial task. Challenges include
internal self-consistency, maintaining specified relationships across
configuration components, potential topological dependencies, and ease of
adaptability across environments (e.g., production and testing).
KaizenFlow uses the
follow design principles for its configuration mechanisms. Some desirable
properties for a configuration layer configuration to have are to be

\begin{itemize}
  \item human-readable and as self-explanatory as possible
  \item version controllable, so that it can be tracked and reproduced
  \item corresponds to a unique system
  \item able to describe every system
  \item modular: a system composed of other systems should be described by the
        composition of the corresponding configurations
  \item executable: a configuration can be executed to generate the
        corresponding system
  \item declarative: the user describes how the system should look and
        other systems can build it
\end{itemize}

% --------------------------------------------------------------------------------
\subsubsection{KaizenFlow configuration layer}
All components in KaizenFlow are configured through a hierarchical data structure
called a \verb|Config|.

A \verb|Config|, together with a builder function, completely characterizes a
KaizenFlow system. In other words, the builders represent the connectivity
between components (which components need to be instantiated and how they need
to be connected), while the \verb|Config| represents the actual parameters needed to
configure the components.

The builder function recursively creates and connects components using other
builders, up to leaf objects, which are directly instantiated.
The \verb|Config| sets parameters in each object that can be accessed and
controlled by the user.

The structure of a concrete Config follows the same organization of the
components inside the KaizenFlow system.
In our implementation parameters for each

% --------------------------------------------------------------------------------
\subsubsection{Example of KaizenFlow configuration layer}
Consider a system called \verb|baz| that contains a component belonging to class
\verb|a|, which in turn includes a component \verb|b|.

\begin{lstlisting}[language=Python, caption=Python example]
class B:

  def __init__(self, param1: List[str], param2: int):
    ...


class FoobarA(AbstractA):

  def __init__(self, val1: int, val2: str, b: B):
    ...


def build_baz_v1(config: Config) -> AbstractA:
  """
  Build a System composed of a class derived from `AbstractA` and a class `B`.
  """
  # Build 'b'.
  b_obj = B(**config["a"]["b_ctor"])
  # Build 'a'.
  if config["a"]["type"] == "a_foobar":
    a_obj = FoobarA(**config["a"]["a_ctor"], b_obj)
  else:
    raise ValueError("Invalid a.type='%s'" % config["a"]["type"])
  return a_obj


config: Config = {
  "a": {
    {
      "type": "a_foobar",
      "a_ctor": {
        "val1": 42,
        "val2": "hello",
        "b_ctor": {
          "param1": ["foo", "bar"],
          "param2": -1,
        }
      }
    }
  }
\end{lstlisting}

This will build a system like:
% TODO(Samarth): Plot the system.

% ###############################################################################
\section{DataPull}

% ================================================================================
\subsection{Data in the real-world}

% --------------------------------------------------------------------------------
\subsubsection{Large variety of data}

Data comes in an extremely large variety and with different characteristics,
for instance:

\begin{itemize}
  \item Different time semantics, e.g.,
        \begin{itemize}
          \item Intervals of data can be \verb|[a, b)| or \verb|(a, b]|
          \item A data bar can be marked at the beginning or at the end of the
                corresponding interval, e.g., \verb|a| or \verb|b|
        \end{itemize}

  \item Structured (e.g., market data) vs unstructured (e.g., alternative
        data)
        \begin{itemize}
          \item Structured data refers to data that is organized in a
                well-defined and predictable format, in tabular nature, where
                data is organized into rows and columns, and each data
                element has a clear data type and schema. Examples
                of structured data include customer names, addresses, purchase
                dates, and product prices.
          \item Unstructured data is data that lacks a specific and organized
                format. Examples are text, images, audio, video, social media
                posts, and more, and it often requires specialized tools and
                techniques, such as natural language processing or machine
                learning, to extract meaningful information from it.

          \item Historical vs real-time
                \begin{itemize}
                  \item The same data can be collected and delivered in batch for a large
                        period of time for research purposes, and then in real-time for
                        production
                  \item In theory, historical and real-time data for the same source
                        should match exactly, this is not always the case, since the
                        historical data can be acquired from different providers or can be
                        lightly post-process to remove some artifacts
                \end{itemize}

          \item Data can be comprised of multiple data sets
                \begin{itemize}
                  \item E.g., in finance there are multiple asset classes (e.g., equities,
                        futures, crypto)
                  \item E.g., in e-commerce, there are customer information, their
                        purchase history, current open order data, marketing data, customer
                        support data
                \end{itemize}

          \item The same data can be delivered at different time resolutions
                \begin{itemize}
                  \item E.g., in finance different data is provided for order book
                        data, trades, minute OHLCV bars, daily bars
                \end{itemize}

          \item Different vendors can provide data for the same data source
                \begin{itemize}
                  \item E.g., Kibot, Binance, CryptoDataDownload provide data for the
                        Binance exchange
                  \item One would expect data from different providers for the same
                        data source to match, yet this is rarely the case due to
                        several reasons, such as different ways of collecting the data,
                        different quality of collection methods
                \end{itemize}

          \item Data and metadata
                \begin{itemize}
                  \item E.g., in e-commerce product inventory can contain product
                        information (e.g., id, name, category, price, quantity in stock)
                        and metadata (e.g., data source, date created, last date modified,
                        schema)
                  \item Some vendors provide metadata associated with data, others don't
                \end{itemize}
        \end{itemize}
\end{itemize}

% ================================================================================
\subsection{DataPull at high-level}

% --------------------------------------------------------------------------------
\subsubsection{DataPull design principles}
The principles that inspire the design of \verb|DataPull| are the following.

\begin{itemize}
  \item Handle data with the same semantics across different sources and types
        of data
  \item Data needs to be versionable
  \item Quality assurance on data needs to be performed as early as possible in
        the system
  \item Historical data needs to be compared and reconciled with real-time data
        on a continuous basis to ensure data sanity
  \item Annotate knowledge timestamps to track when the data was available
        at different parts of the processing pipeline
  \item Abstract storage formats (e.g., Parquet, CSV, databases) and data modes
        (e.g., historical vs real-time) so that clients are not affected by
        these changes
  \item Capture and store data exactly as-is without any change, not even to
        fix obvious issues
  \item Post processing (e.g., remove malformed data, deduplicate data) needs
        to be easy to perform on the fly while serving data
\end{itemize}

% --------------------------------------------------------------------------------
\subsubsection{ETtL approach}
\verb|DataPull| employs a variation of the ETL approach, called EtLT (i.e.,
extract, lightly transform, load, transform) for downloading both data and
metadata.

Data is extracted from an external data source, lightly transformed, and
then saved into permanent storage.

The light transform should be kept non-destructive (e.g., reshaping the data)
since we want to capture the data as-is to be able to collect data as close as
possible to what comes from the wire.

Downstream data pipelines read the
data with a client interface, standardized in terms of timing semantic
and format of the data.

In general there are separate pipelines for data and metadata for the
same data source.

% --------------------------------------------------------------------------------
\subsubsection{Extract stage}

The goal is to acquire raw data from an external source and archive it into a
permanent storage backend (e.g., file-system and / or database). The data can
be either historical or real-time. `DataPull` typically doesn't process the
data at all, but rather we prefer to save the data raw as it comes from the
wire.

% --------------------------------------------------------------------------------
\subsubsection{Transform stage}

Typically, `DataPull` prefers to load the data in the
backend with minor or no transformation. Specifically we allow changing the
representation of
the data / format (e.g., removing some totally useless redundancy, compressing
the data, transforming from strings to datetimes). We don't allow changing the
semantics or filter columns. This is done dynamically in the `client` stage

% --------------------------------------------------------------------------------
\subsubsection{Load stage}

The load stage simply saves the data into one of the supported backends.

Typically, we prefer to save:
\begin{itemize}
  \item Historical data into Parquet format since it supports more naturally the
        access patterns needed for long simulations
  \item Real-time data into a database since this makes it easy to append and retrieve
        data in real-time. Often we want to also append real-time data to Parquet
\end{itemize}

% --------------------------------------------------------------------------------
\subsubsection{Client stage}

The client stage allows downstream pipelines to access data from the backend
storage.

The access pattern from a downstream pipeline follows along the line of
"from the dataset \verb|DATA|
get the features \verb|XYZ|
for entities \verb|ABC|
in the period \verb|(..., ...]|".

We prefer to perform some transformations that are lightweight (e.g., converting
Unix epochs in timestamps) or still evolving (e.g., understanding the timing
semantic of the data) within the client stage, rather than in the transform stage,
so that they can easily be updated and iterated.

The client is implemented in two stages:

The first stage called \verb|VendorDataReader|
\begin{itemize}
  \item This layer is vendor and data set specific
  \item Handle all the peculiarities in format and semantic of a specific
        vendor
\end{itemize}

The second stage is called \verb|DataReader|

\begin{itemize}
  \item This layer is independent of the vendor
  \item Implement behaviors that are orthogonal to vendors, such as

        \begin{itemize}
          \item Streaming/real-time or batch/historical
          \item Time-stitching of streaming/batch data, i.e.,``overlap'' multiple
                data sources giving a single view of the data

                \begin{itemize}
                  \item E.g., the data from the last day comes from a real-time source
                        while the data before that comes from an historical source
                \end{itemize}
          \item Replaying, i.e., serialize the data to disk and read it back,
                implementing as-of-time semantic based on knowledge time

                \begin{itemize}
                  \item This behavior is orthogonal to streaming/batch and stitching,
                        i.e., one can replay any \texttt{MarketData}, including an already
                        replayed one
                \end{itemize}
          \item Data is accessed based on intervals \verb|[start_timestamp, end_timestamp]|
                using different open/close semantics, but always preventing future peeking
          \item Support real-time behaviors, such as knowledge time, wall clock time,
                and blocking behaviors (e.g., ``is the last data available?'')
          \item Handles desired timezone for timestamps
        \end{itemize}
\end{itemize}

%```mermaid
%flowchart
%  Vendor Data --> VendorDataReader --> DataReader --> User
%```

% --------------------------------------------------------------------------------
\subsubsection{Supported storage backend}

\verb|DataPull| supports multiple storage backends:

\begin{itemize}
  \item database (e.g., Postgres, MongoDB)
  \item local filesystem
  \item remote backends (e.g., AWS S3 bucket)
\end{itemize}

% --------------------------------------------------------------------------------
\subsubsection{Supported data format}

Data can be saved on filesystems in different formats (e.g., CSV, JSON,
Parquet).

\textbf{Data formats}. The main data formats that \verb|DataPull|
supports are:

\begin{itemize}
  \item CSV
        \begin{itemize}

          \item Pros
                \begin{itemize}
                  \item Easy to inspect
                  \item Easy to load / save
                  \item Everybody understands it
                \end{itemize}

          \item Cons
                \begin{itemize}
                  \item Data can't be easily sliced by asset ids / by time
                  \item Large footprint (non-binary), although it can be compressed on the
                        fly (e.g., \verb|.csv.gz|)
                \end{itemize}
        \end{itemize}

  \item Parquet
        \begin{itemize}

          \item Pros
                \begin{itemize}
                  \item Compressed
                  \item AWS friendly
                  \item Data can be easily sliced by asset ids and time
                \end{itemize}

          \item Cons
                \begin{itemize}
                  \item Not easy to inspect

                        \begin{itemize}
                          \item Solution: use wrapper to convert to CSV
                        \end{itemize}
                  \item Difficult to append

                        \begin{itemize}
                          \item Solution: use chunking + defragmentation
                        \end{itemize}
                  \item Cumbersome for real-time data
                \end{itemize}
        \end{itemize}

  \item Database
        \begin{itemize}

          \item Pros
                \begin{itemize}
                  \item Easy to inspect
                  \item Support any access pattern
                  \item Friendly for real-time data
                \end{itemize}

          \item Cons
                \begin{itemize}
                  \item Need devops to manage database instance
                  \item Difficult to track lineage and version
                \end{itemize}
        \end{itemize}
\end{itemize}

Unfortunately there is not an obvious best solution, so \verb|DataPull|
supports multiple representations and converting data between them.

% --------------------------------------------------------------------------------
\subsubsection{Adapters}

- Adapters to different data sources
% from ?

% ================================================================================
\subsection{DataPull data format}
% from datapull.explanation.md

We use the following invariants when storing data during data on-boarding and
processing:

\begin{itemize}
  \item Data quantities are associated either to intervals \verb|[a, b)|
        (e.g., ``the price return over the interval \verb|[a, b)|'')
        or to a single point in time (e.g., ``the close price at 9am UTC'')
  \item Every piece of data is labeled with the end of the sampling interval
        since that is a better lower bound for the knowledge time
        (e.g., for a quantity computed in an interval
        \verb|[06:40:00, 06:41:00)|, the event timestamp is \verb|06:41:00|),
        or with the point in time.
  \item Timestamps are always time-zone aware, typically using UTC timezone
  \item Every piece of data has a knowledge timestamp (aka ``as-of-date'')
        which represent when a component in \verb|KaizenFlow| was made aware of it
        according to a wall-clock:
        \begin{itemize}
          \item In general multiple timestamps associated with different events
                can be tracked (e.g., \verb|start_download_timestamp| to track
                when the download of data started and \verb|end_download_timestamp|
                when the download was actually completed)
          \item No component or system should depend on data available strictly
                before the knowledge timestamp
        \end{itemize}
  \item Data is versioned: every time we modify the schema or the semantics of
        the data, the version of the data should be bumped up using semantic
        versioning and a changelog should be updated to track the changes
\end{itemize}

% ================================================================================
\subsection{Data layout}

\verb|DataPull| keeps data bundled together by execution run instead of by data
element.

E.g., assume we run a flow called \verb|XYZ_sanity_check| every day
and the flow generates three pieces of data, one file \verb|output.txt| and two
directories \verb|logs|, \verb|temp_data|.

\verb|DataPull| organizes the data in a directory structure like:

\begin{verbatim}
- XYZ_sanity_check/
  - run.{date}/
    - output.txt
    - logs/
    - temp_data/
  - run.{date}.manual/
    - output.txt
    - logs/
    - temp_data/
\end{verbatim}

which we consider superior to the alternative data layout below:

\begin{verbatim}
- XYZ_sanity_check/
  - output.{date}/
    - output.txt
  - logs.{date}/
  - temp_data.{date}/
    ...
\end{verbatim}

The reasons why the first data layout is superior are:

\begin{enumerate}
  \item It's easier to delete a single run by deleting a single dir instead of
        deleting multiple files
  \item It allows the format of the data to evolve over time without having to
        change the schema of the data retroactively
  \item It allows scripts post-processing the data to point to a directory
        with a specific run and work out of the box
  \item it's easier to move the data for a single run from one dir (e.g.,
        locally) to another (e.g., a central location) in one command
  \item there is redundancy and visual noise, e.g., the same data is
        everywhere
\end{enumerate}

We can tag directory by a run mode (e.g., \verb|manual| vs \verb|scheduled|) by
adding the proper suffix to a date-dir.

% ================================================================================
\subsection{Data/timing semantic}
% from datapull.explanation.md

\subsubsection{Data set naming scheme}\label{data-set-naming-scheme}

Each data set is stored in a data lake with a path and name that
describe its metadata according to the following signature:

\begin{verbatim}
dataset_signature={download_mode}.{downloading_entity}.{action_tag}.{data_format}.{data_type}.{asset_type}.{universe}.{vendor}.{exchange_id}.{version\[-snapshot\]}.{extension}
\end{verbatim}

TODO(gp): @juraj add a \{backend\} = s3, postgres, mongo, local\_file

The signature schema might be dependent on the backend

E.g.,

\begin{verbatim}
bulk/airflow/downloaded_1min/csv/ohlcv/futures/universe_v1_0/ccxt/binance/v1_0-20220210/BTC_USD.csv.gz
\end{verbatim}

We use \verb|-| to separate pieces of the same attribute (e.g.,
version and snapshot) and \verb|_| as replacements of a space
character.

The organization of files in directories should reflect the naming
scheme. We always use one directory per attribute for files (e.g.,
\verb|bulk.airflow.csv/...| or \verb|bulk/airflow/csv/...|). When
the metadata is used not to identify a file in the filesystem (e.g., for
a script or as a tag) then we use \verb|.| as separators between the
attributes.

\subsubsection{Data set attributes}\label{data-set-attributes}

There are several ``attributes'' of a data set:

\begin{itemize}
  \item \verb|download_mode|: the type of downloading mode

        \begin{itemize}
          \item \verb|bulk|

                \begin{itemize}
                  \item Aka ``one-shot'', ``one-off'', and improperly ``historical''
                  \item Data downloaded in bulk mode, as one-off documented operations
                  \item Sometimes it's referred to as ``historical'', since one downloads
                        the historical data in bulk before the real-time flow is deployed
                \end{itemize}

          \item \verb|periodic|
                \begin{itemize}
                  \item Aka ``scheduled'', ``streaming'', ``continuous'', and improperly
                        ``real-time''
                  \item Data is captured regularly and continuously
                  \item Sometimes it's referred as to ``real-time'' since one capture this
                        data
                  \item It can contain information about the frequency of downloading
                        (e.g., \verb|periodic-5mins|, \verb|periodic-EOD|) if it needs
                        to be identified with respect to others
                \end{itemize}

          \item \verb|unit_test|
                \begin{itemize}
                  \item Data used for unit test (independently if it was downloaded
                        automatically or created manually)
                \end{itemize}
        \end{itemize}

  \item \verb|downloading_entity|: different data depending on whom
        downloaded it, e.g.,

        \begin{itemize}
          \item \verb|airflow|: data was downloaded as part of the automatic flow
          \item \verb|manual|: data download was triggered manually (e.g., running
                the download script)
        \end{itemize}
  \item \verb|action_tag|: information about the downloading, e.g.,
        \verb|downloaded_1min| or \verb|downloaded_EOD|
  \item \verb|data_format|: the format of the data, e.g.,

        \begin{itemize}
          \item \verb|csv| (always csv.gz, there is no reason for not compressing
                the data)
          \item \verb|parquet|
        \end{itemize}
  \item \verb|data_type|: what type of data is stored, e.g.,

        \begin{itemize}
          \item \verb|ohlcv|, \verb|bid_ask|, \verb|market_depth| (aka
                \verb|order_book|), \verb|bid_ask_market_data| (if it
                includes both), \verb|trades|
        \end{itemize}
  \item \verb|asset_type|: what is the asset class

        \begin{itemize}
          \item E.g., futures, spot, options
        \end{itemize}
  \item \verb|universe|: the name of the universe containing the possible
        assets

        \begin{itemize}
          \item Typically, the universe can have further characteristics and it can
                be also versioned
          \item E.g., \verb|universe_v1_7|
        \end{itemize}
  \item \verb|vendor|: the source that provided the data

        \begin{itemize}
          \item Aka ``provider''
          \item E.g., \verb|ccxt|, \verb|crypto_chassis|,
                \verb|cryptodata_download|, \verb|kaiko|,
          \item Data can also be downloaded directly from an exchange (e.g.,
                \verb|coinbase|, \verb|binance|)
        \end{itemize}
  \item \verb|exchange_id|: which exchange the data refers to

        \begin{itemize}
          \item E.g., \verb|binance|
        \end{itemize}
  \item \verb|version|: any data set needs to have a version

        \begin{itemize}
          \item Version is represented as major, minor, patch according to semantic
                versioning in the format \verb|v{a}_{b}_{c}| (e.g.,
                \verb|v1_0_0|)
          \item If the schema of the data is changed the major version is increased
          \item If a bug is fixed in the downloader that improves the semantic of
                the data, but it's not a backward incompatible change, the minor
                version is increased
          \item The same version can also include an optional \verb|snapshot|
                which refers to the date when the data was downloaded (e.g., a
                specific date \verb|20220210| to represent when the day on which
                the historical data was downloaded, i.e., the data was the
                historical data as-of 2022-02-10)
          \item Note that \verb|snapshot| and \verb|version| have an overlapping
                but not identical meaning. \verb|snapshot| represents when the
                data was downloaded, while \verb|version| refers to the evolution
                of the semantic of the data and of the downloader. E.g., the same
                data source can be downloaded manually on different days with the
                same downloader (and thus with the same version).
        \end{itemize}
  \item \verb|asset_type|: which cryptocurrency the data refers to:

        \begin{itemize}
          \item Typically, there is one file per asset (e.g.,
                \verb|BTC_USDT.csv.gz|)
          \item Certain data formats can organize the data in a more complex way

                \begin{itemize}
                  \item E.g., Parquet files save the data in a directory structure
                        \verb|{asset}_{year}_{month\}_data.parquet|
                \end{itemize}
        \end{itemize}
\end{itemize}

It is possible that a single data set covers multiple values of a
specific attribute

\begin{itemize}
  \item E.g., a data set storing data for both futures and spot, can have
        \verb|asset_type=futures_spot|
\end{itemize}

Not all the cross-products are possible, e.g.

\begin{itemize}
  \item there is no data set with \verb|download_mode=periodic| scheduled
        by Airflow and \verb|downloading_entity=manual|
\end{itemize}

We organize the schema in terms of access patterns for the modeling and
analysis stage

\begin{itemize}
  \item E.g., \verb|snapshot| comes before \verb|vendor| since in
        different snapshots we can have different universes
  \item E.g., snapshot -\textgreater{} dataset -\textgreater{} vendor
        -\textgreater{} exchange -\textgreater{} coin
  \item A universe is just a mapping of a tag (e.g., v5) to a set of
        directories
\end{itemize}

Each data set has multiple columns.

% ================================================================================
\subsection{KaizenFlow time series databases}

% --------------------------------------------------------------------------------
\subsubsection{Time series databaes}
Data stores supporting stream computing are specialized databases
optimized for storing and serving time series data—data points indexed in time
order. Time series data consists of sequences of values or events
collected at regular or irregular intervals over time. These databases are
designed to handle the unique challenges posed by time series data, including
large volumes of data and real-time processing.

Some key aspects of time series databases:

\subsubsection{Time-indexed data time as a primary key} In time series
databases, time is the primary axis, meaning that data is organized based
on timestamps. The data usually consists of sequences of measurements taken
over time, such as sensor data, stock market data, or server metrics.

\subsubsection{High-performance write and read efficient data ingestion}
Time series are optimized for fast writes, as time series data often involves
high-frequency data collection (e.g., IoT sensors, financial tick data).

\subsubsection{Optimized query performance}
Time series databases are also optimized for time-based queries, such as
aggregations over time intervals or retrieving data points within a specific
time range.

% --------------------------------------------------------------------------------
\subsubsection{KaizenFlow DB}

- knowledge time
- different views of the data spliced together (e.g., historical and real-time)

% This section is about the properties for time series/temporal databases.

% --------------------------------------------------------------------------------
\subsubsection{Software architecture}

% ImClient, MarketData

% ###############################################################################
\section{DataFlow}

% ================================================================================
\subsection{Computation as graphs}
% from docs/dataflow/computation_as_graphs.explanation.md

% --------------------------------------------------------------------------------
\subsubsection{DataFlow framework}
DataFlow is a computing framework to build and test AI and machine learning
models that can run:
\begin{itemize}
  \item with no changes in batch vs streaming mode
  \item in different modes, which trade off timing accuracy and speed (timed,
        non-timed, replayed simulation, and real-time execution)
\end{itemize}

The working principle underlying DataFlow is to run a model in terms of time
slices of data so that both batch/historical and streaming/real-time semantics
can be accommodated without any change in the model description.

Some of the advantages of the DataFlow approach are:
\begin{itemize}
  \item Adapt a procedural description of a model to a reactive/streaming semantic
  \item Tiling to fit in memory
  \item Cached computation
  \item A clear timing semantic which includes support for knowledge time and
        detection of future peeking
  \item Ability to replay and debug model executions
\end{itemize}

A DAG Node has:
\begin{itemize}
  \item several inputs
  \item several outputs
  \item a unique node id (aka \verb|nid|)
  \item a (optional) state
\end{itemize}

Inputs and outputs to a DAG Node are dataframes, represented in the current
implementation as \verb|Pandas| dataframes.
DAG node uses the inputs to compute the output (e.g., using \verb|Pandas| and
\verb|Sklearn| libraries).
A DAG node can execute in multiple ``phases'', referred to through the
corresponding methods called on the DAG (e.g., \verb|fit|, \verb|predict|,
\verb|save_state|, \verb|load_state|).

A DAG node stores an output value for each output and method name.

TODO(Samarth): circle with inputs and outputs

% --------------------------------------------------------------------------------
\subsubsection{DAG node examples}
Examples of operations that may be performed by nodes include:

\begin{itemize}
  \item Loading data (e.g., market or alternative data)
  \item Resampling data bars (e.g., OHLCV data, tick data in finance)
  \item Computing rolling average (e.g., TWAP/VWAP, volatility of returns)
  \item Adjusting returns by volatility
  \item Applying EMAs (or other filters) to signals
  \item Performing per-feature operations, each requiring multiple features
  \item Performing cross-sectional operations (e.g., factor residualization,
        Gaussian ranking)
  \item Learning/applying a machine learning model (e.g., using sklearn)
  \item Applying custom (user-written) functions to data
\end{itemize}

Further examples include nodes that maintain relevant trading state, or
that interact with an external environment:

\begin{itemize}
  \item Updating and processing current positions
  \item Performing portfolio optimization
  \item Generating trading orders
  \item Submitting orders to an API
\end{itemize}

% --------------------------------------------------------------------------------
\subsubsection{DataFlow model}
A DataFlow model (aka \verb|DAG|) is a direct acyclic graph composed of
DataFlow nodes

It allows one to connect, query the structure, \ldots{}

Running a method on a DAG means running that method on all its nodes in
topological order, propagating values through the DAG nodes.

TODO(Paul, Samarth): Add picture.

% --------------------------------------------------------------------------------
\subsubsection{DagConfig}
A \verb|Dag| can be built by assembling Nodes using a function representing the
connectivity of the nodes and parameters contained in a \verb|Config| (e.g.,
through a call to a builder \verb|DagBuilder.get_dag(config)|).

A DagConfig is hierarchical and contains one subconfig per DAG node. It should
only include \verb|Dag| node configuration parameters, and not information
about \verb|Dag| connectivity, which is specified in the \verb|Dag| builder
part.

% --------------------------------------------------------------------------------
\subsection{Dataframe as unit of computation}
The basic unit of computation of each node is a ``dataframe''. Each node
takes multiple dataframes through its inputs, and emits one or more
dataframes as outputs.

In mathematical terms, a dataframe can be described as a two-dimensional
labeled data structure, similar to a matrix but with more flexible features.

A dataframe $df$ can be represented as:

$$
  \df = \left[ \begin{array}{cccc}
      a_{11} & a_{12} & \cdots & a_{1n} \\
      a_{21} & a_{22} & \cdots & a_{2n} \\
      \vdots & \vdots & \ddots & \vdots \\
      a_{m1} & a_{m2} & \cdots & a_{mn} \\
    \end{array} \right]
$$

where:
\begin{itemize}
  \item $m$ is the number of rows (observations).
  \item $n$ is the number of columns (variables).
  \item $a_{ij}$ represents the element of the Dataframe in the $i$-th row
        and $j$-th column.
\end{itemize}

Some characteristics of dataframes are:
\begin{enumerate}
  \item Labeled axes:
        \begin{itemize}
          \item Rows and columns are labeled, typically with strings, but labels can
                be of any hashable type.
          \item Rows are often referred to as indices and columns as column headers.
        \end{itemize}
  \item Heterogeneous data types:
        \begin{itemize}
          \item Each column $j$ can have a distinct data type, denoted as
                $dtype_j$
          \item Common data types include integers, floats, strings, and datetime
                objects.
        \end{itemize}
  \item Mutable size:
        \begin{itemize}
          \item Rows and columns can be added or removed, meaning the size of
                $df$ is mutable.
          \item This adds to the flexibility as compared to traditional matrices.
        \end{itemize}
  \item Alignment and operations:
        \begin{itemize}
          \item Dataframes support alignment and arithmetic operations along rows
                and columns.
          \item Operations are often element-wise but can be customized with
                aggregation functions.
        \end{itemize}
  \item Missing Data Handling:
        \begin{itemize}
          \item Dataframes can contain missing data, denoted as $\NaN$ or a
                similar placeholder.
          \item They provide tools to handle, fill, or remove missing data.
        \end{itemize}
  \item Multidimensionality:
        \begin{itemize}
          \item Tensor-like objects are supported through row or column
                ``multi-indices".
          \item If time is the primary key, then multi-index columns can be used
                to support panel or higher-dimensional data at each timestamp.
        \end{itemize}
\end{enumerate}

% ================================================================================
\subsection{DAG execution}

% --------------------------------------------------------------------------------
\subsubsection{Simulation kernel}

A computation graph is a directed graph where nodes represent operations or
variables, and edges represent dependencies between these operations.

For example, in a computation graph for a mathematical expression, nodes would
represent operations like addition or multiplication, while edges would
indicate the order (and grouping) of operations.

The KaizenFlow simulation kernel schedules nodes according to their
dependencies.

% --------------------------------------------------------------------------------
\subsection{Details about simulation kernel}

The most general case of simulation consists of multiple nested loops:

\begin{enumerate}
  \item \textbf{Multiple DAG computation}. The general workload contains multiple
        DAG computations, each one inferred through a \verb|Config| belonging to
        a list of \verb|Config|s describing the entire workload to execute.
        \begin{itemize}
          \item In this set-up each DAG computation is independent, although
                some pieces of computations can be common across the workload.
                KaizenFlow will compute and then cache the common computations
                automatically as part of the framework execution
        \end{itemize}

  \item \textbf{Learning pattern}. For each DAG computation, multiple train/predict loops represent
        different machine learning patterns (e.g., in-sample vs out-of-sample,
        cross-validation, rolling window)
        \begin{itemize}
          \item This loop accommodates the need for nodes with state to be
                driven to learn parameters and hyperparameters and then use the
                learned state to predict on unseen data (i.e., out-of-sample)
        \end{itemize}
  \item \textbf{Temporal tiling}. Each DAG computation runs over a tile
        representing an interval of time
        \begin{itemize}
          \item As explained in section XYZ, KaizenFlow partition the time
                dimension in multiple tiles
          \item Temporal tiles might overlap to accommodate the amount of
                memory needed by each node (see XYZ), thus each timestamp will be
                covered by at least one tile. In the case of DAG nodes with no
                memory, then time is partitioned in non-overlapping tiles.
          \item The tiling pattern over time does not affect the result as long
                as the system is properly designed (see XYZ)
        \end{itemize}
  \item \textbf{Spatial tiling}. Each temporal slice can be computed in terms
        of multiple sections across the horizontal dimension of the dataframe
        inputs, as explained in section XYZ.

        \begin{itemize}
          \item This is constrained by nodes that compute features
                cross-sectionally, which require the entire space slice to be
                computed at once
        \end{itemize}
  \item \textbf{Single DAG computation}. Finally a topological sorting
        based on the specific DAG connectivity is performed in order to execute
        nodes in the proper order. Each node executes over temporal and spatial tiles.
\end{enumerate}

TODO(gp): Add picture showing the various loops

Note that it is possible to represent all the computations from the above loops
in a single ``scheduling graph'' and use this graph to schedule executions in a
global fashion.

Parallelization across CPUs comes naturally from the previous approach,
since computations that are independent in the scheduling graph can be
executed in parallel, as described in Section XYZ.

Incremental and cached computation is built-in in the scheduling
algorithm since it's possible to memoize the output by checking for a
hash of all the inputs and of the code in each node, as described in Section XYZ.

Even though each single DAG computation is required to have no loops, a System
(see XYZ) can have components introducing loops in the computation
(e.g., a Portfolio component in a trading system, where a DAG computes
forecasts which are acted upon based on the available funds). In this
case, the simulation kernel needs to enforce dependencies in the time
dimension.

% --------------------------------------------------------------------------------
\subsection{Nodes ordering for execution}

TODO(gp, Paul): Extend this to the multiple loop.

Topological sorting is a linear ordering of the vertices of a directed
graph such that for every directed edge from vertex u to vertex v, u
comes before v in the ordering. This sorting is only possible if the
graph has no directed cycles, i.e., it must be a Directed Acyclic Graph
(DAG).

\begin{lstlisting}[language=Python]
def topological_sort(graph):
    visited = set()
    post_order = []

    def dfs(node):
        if node in visited:
            return
        visited.add(node)
        for neighbor in graph.get(node, []):
            dfs(neighbor)
        post_order.append(node)

    for node in graph:
        dfs(node)

    return post_order[::-1]  # Reverse the post-order to get the topological order
\end{lstlisting}

% --------------------------------------------------------------------------------
\subsection{Heuristics for splitting code in nodes}

There are degrees of freedom in splitting the work between various nodes
of a graph E.g., the same DataFlow computation can be described with
several nodes or with a single node containing all the code

The trade-off is often between several metrics:

\begin{itemize}
  \item Observability

        \begin{itemize}
          \item More nodes make it easier to:

                \begin{itemize}
                  \item observe and debug intermediate the result of complex computation
                  \item profile graph executions to understand performance bottlenecks
                \end{itemize}
        \end{itemize}
  \item latency/throughput

        \begin{itemize}
          \item More nodes:

                \begin{itemize}
                  \item allows for better caching of computation
                  \item allows for smaller incremental computation when only one part of
                        the inputs change
                  \item prevents optimizations performed across nodes
                  \item incurs in more simulation kernel overhead for scheduling
                  \item allows more parallelism between nodes being extracted
                        and exploited
                \end{itemize}
        \end{itemize}
  \item memory consumption

        \begin{itemize}
          \item More nodes:

                \begin{itemize}
                  \item allows one to partition the computation in smaller
                        chunks requiring less working memory
                \end{itemize}
        \end{itemize}
\end{itemize}

A possible heuristic is to start with smaller nodes, where each node
has a clear function, and then merge nodes if this is shown to improve
performance

%\begin{tikzpicture}
%
%  \def \n {5}
%  \def \radius {3cm}
%  \def \margin {8} % margin in angles, depends on the radius
%
%  \foreach \s in {1,...,\n}
%  {
%  \node[draw, circle] at ({360/\n * (\s - 1)}:\radius) {$\s$};
%  \draw[->, >=latex] ({360/\n * (\s - 1)+\margin}:\radius)
%  arc ({360/\n * (\s - 1)+\margin}:{360/\n * (\s)-\margin}:\radius);
%  }
%\end{tikzpicture}
%
%
%\definecolor{c1a1a1a}{RGB}{26,26,26}
%\def \globalscale {1.000000}
%\begin{tikzpicture}[y=1cm, x=1cm, yscale=\globalscale,xscale=\globalscale, every node/.append style={scale=\globalscale}, inner sep=0pt, outer sep=0pt]
%  \path[line width=0.0265cm] (1.5692, 26.5331) rectangle (4.736, 24.3363);
%  \path[draw=c1a1a1a,even odd rule,line width=0.0365cm] (3.3951, 26.419) rectangle (7.3893, 23.9939);
%  \path[draw=c1a1a1a,even odd rule,line width=0.0365cm] (7.6776, 26.4673) rectangle (12.4688, 24.0429);
%\end{tikzpicture}
%
%
%\tikzset{every picture/.style={line width=0.75pt}} %set default line width to 0.75pt
%
%\begin{tikzpicture}[x=0.75pt,y=0.75pt,yscale=-1,xscale=1]
%  %uncomment if require: \path (0,300); %set diagram left start at 0, and has height of 300
%
%  %Shape: Rectangle [id:dp08954756900345329]
%  \draw   (100,109) -- (170,109) -- (170,149) -- (100,149) -- cycle ;
%  %Shape: Rectangle [id:dp39153509095750505]
%  \draw   (202,108) -- (272,108) -- (272,148) -- (202,148) -- cycle ;
%\end{tikzpicture}
%
%\tikzset{every picture/.style={line width=0.75pt}} %set default line width to 0.75pt
%
%\begin{tikzpicture}[x=0.75pt,y=0.75pt,yscale=-1,xscale=1]
%  %uncomment if require: \path (0,300); %set diagram left start at 0, and has height of 300
%
%  %Shape: Circle [id:dp6063710096356814]
%  \draw  [line width=1.5]  (100,144) .. controls (100,130.19) and (111.19,119) .. (125,119) .. controls (138.81,119) and (150,130.19) .. (150,144) .. controls (150,157.81) and (138.81,169) .. (125,169) .. controls (111.19,169) and (100,157.81) .. (100,144) -- cycle ;
%  %Straight Lines [id:da5691059006793302]
%  \draw    (100,103) -- (108.14,120.19) ;
%  \draw [shift={(109,122)}, rotate = 244.65] [color={rgb, 255:red, 0; green, 0; blue, 0 }  ][line width=0.75]    (10.93,-3.29) .. controls (6.95,-1.4) and (3.31,-0.3) .. (0,0) .. controls (3.31,0.3) and (6.95,1.4) .. (10.93,3.29)   ;
%  %Straight Lines [id:da584136245408913]
%  \draw    (144,101) -- (137.69,118.12) ;
%  \draw [shift={(137,120)}, rotate = 290.22] [color={rgb, 255:red, 0; green, 0; blue, 0 }  ][line width=0.75]    (10.93,-3.29) .. controls (6.95,-1.4) and (3.31,-0.3) .. (0,0) .. controls (3.31,0.3) and (6.95,1.4) .. (10.93,3.29)   ;
%  %Straight Lines [id:da5185654683072161]
%  \draw    (116,168) -- (109.69,185.12) ;
%  \draw [shift={(109,187)}, rotate = 290.22] [color={rgb, 255:red, 0; green, 0; blue, 0 }  ][line width=0.75]    (10.93,-3.29) .. controls (6.95,-1.4) and (3.31,-0.3) .. (0,0) .. controls (3.31,0.3) and (6.95,1.4) .. (10.93,3.29)   ;
%  %Straight Lines [id:da9737739663255534]
%  \draw    (137,167) -- (145.14,184.19) ;
%  \draw [shift={(146,186)}, rotate = 244.65] [color={rgb, 255:red, 0; green, 0; blue, 0 }  ][line width=0.75]    (10.93,-3.29) .. controls (6.95,-1.4) and (3.31,-0.3) .. (0,0) .. controls (3.31,0.3) and (6.95,1.4) .. (10.93,3.29)   ;
%
%  % Text Node
%  \draw (115,100) node [anchor=north west][inner sep=0.75pt]   [align=left] {...};
%  % Text Node
%  \draw (106,135) node [anchor=north west][inner sep=0.75pt]   [align=left] {Node};
%  % Text Node
%  \draw (122,171) node [anchor=north west][inner sep=0.75pt]   [align=left] {...};
%  % Text Node
%  \draw (100,81) node [anchor=north west][inner sep=0.75pt]   [align=left] {Inputs};
%  % Text Node
%  \draw (101,196) node [anchor=north west][inner sep=0.75pt]   [align=left] {Outputs};
%\end{tikzpicture}
%
%
%
%\digraph{abc}{
%  rankdir=LR;
%  a -> b -> c;
%}

% ================================================================================
\subsection{DataFlow data format}
% from docs/dataflow/dataflow_data_format.explanation.md

As explained in XYZ, raw data from \verb|DataPull| is stored in a
``long format'', where the data is conditioned on the asset (e.g.,
\verb|full_symbol|), e.g.,

\begin{verbatim}
                             full_symbol        open    high    low     close ...
timestamp
2021-09-01 00:00:00+00:00   binance::ADA_USDT   2.768   2.770   2.762   2.762
2021-09-01 00:00:00+00:00   binance::AVAX_USDT  39.510  39.540  39.300 39.320
2021-09-01 00:00:00+00:00   binance::ADA_USDT   2.763   2.765   2.761   2.764
\end{verbatim}

\verb|DataFlow| represents data through multi-index dataframes, where:

\begin{itemize}
  \item the index is a full timestamp
  \item the outermost column index is the ``feature''
  \item the innermost column index is the asset, e.g.,
\end{itemize}

\begin{verbatim}
                                                            close           high
                           binance::ADA_USDT   binance::AVAX_USDT            ...
timestamp
2021-09-01 00:00:00+00:00              2.762                39.32
2021-09-01 00:00:00+00:00              2.764                39.19
\end{verbatim}

The reason for this convention is that typically features are computed
in a uni-variate fashion (e.g., asset by asset), and DataFlow can
vectorize computation over the assets by expressing operations in terms
of the features. E.g., we can express a feature as

\begin{lstlisting}[language=Python]
df["close", "open"].max() - df["high"]).shift(2)
\end{lstlisting}

A user can work with DataFlow at 4 levels of abstraction:

\begin{enumerate}

  \item Pandas long-format (non multi-index) dataframes and for-loops
        \begin{itemize}
          \item We can do a group-by or filter by \verb|full_symbol|
          \item Apply the transformation on each resulting dataframe
          \item Merge the data back into a single dataframe with the long-format
        \end{itemize}

  \item Pandas multiindex dataframes
        \begin{itemize}
          \item The data is in the DataFlow native format
          \item We can apply the transformation in a vectorized way
          \item This approach is best for performance and with compatibility with
                DataFlow point of view
          \item An alternative approach is to express multi-index transformations in
                terms of approach 1 (i.e., single asset transformations and then
                concatenation). This approach is functionally equivalent to a
                multi-index transformation, but typically slow and memory
                inefficient
        \end{itemize}

  \item DataFlow nodes
        \begin{itemize}
          \item A DataFlow node implements certain transformations on dataframes
                according to the DataFlow convention and interfaces
          \item Nodes operate on the multi-index representation by typically calling
                functions from level 2 above
        \end{itemize}

  \item DAG
        \begin{itemize}
          \item A series of transformations in terms of DataFlow nodes
        \end{itemize}
\end{enumerate}

%An example ./amp/dataflow/notebooks/gallery_dataflow_example.ipynb
%TODO(gp): Fix this reference.

% ================================================================================
\subsection{KaizenFlow System}

% --------------------------------------------------------------------------------
\subsubsection{Motivation}

While KaizenFlow requires a DAG should not have cycles, general computing
systems might need to reuse the state from computation performed on past data.
E.g., in a trading system, there is often a Forecast component that can be
modeled as a DAG with no cycles and a Portfolio object that uses the forecasts
to compute the desired allocation of capital across different positions based on
the previous positions.

KaizenFlow supports this need by assembling multiple DAGs into a complete
\verb|System| that allows cycles.

The assumption is that DAGs are computationally expensive, while other components mainly execute light procedural computation that requires interaction with external
objects such as databases, filesystems, sockets.

TODO(gp): Add picture

TODO(gp): Explain that System are derived from other Python objects.

% ================================================================================
\subsection{Timing semantic and clocks}
% from all.timing_semantic_and_clocks.md

% ================================================================================
\subsection{Batch and streaming mode using tiling}
% from all.batch_and_streaming_mode_using_tiling.explanation.md

% --------------------------------------------------------------------------------
\subsubsection{The property of tilability}

The working principle of a DataFlow computation is that nodes should be
able to compute their outputs from their inputs without a dependency on
how the inputs are partitioned along the dataframe axes of the inputs
(e.g., the time and the feature axes). When this property is valid we
call a computation ``tilable''.

A slightly more formal definition is that a computation $f()$ is
tilable if: \[f(dfX \cup dfY) = f(dfX) \cup f(dfY)\] where:

\begin{itemize}
  \item $dfX$ and $dfY$ represent input data frames (which can optionally)
        overlap
  \item $\cup$ is an operation of concat along consecutive
  \item $f()$ is a node operation
\end{itemize}

TODO(gp): In reality the property requires that feeding data, computing,
and then filtering is invariant, like

f(A, B)

\[\forall t1 \le t2, t3 \le t4: \exists T:
  f(A[t1 - T:t2] \cup A[t3 - T:t4])[t1:t4]
  = f(A[t1 - T:t2])[t1:t4] \cup f(A[t3 - T:t4])[t1:t4]
\]

This property resembles linearity in the sense that a transformation
$f()$ is invariant over the partitioning of the data.

A sufficient condition for a DAG computation to be tileable, is for all
the DAG nodes to be tileable. The opposite is not necessarily true, and
no interest in general, since we are interested in finding ways to
describe the computation so that it is tileable.

A node that has no memory, e.g., whose computation

Nodes can have ``memory'', where the output for a given tile depends on
previous tile. E.g., a rolling average has memory since samples with
different timestamps are combined together to obtain the results. A node
with finite memory is always tileable, while nodes with infinite memory
are not necessarily tileable. If the computation can be expressed in a
recursive form across axes (e.g., an exponentially weighted moving
average for adjacent intervals of times), then it can be made tileable
by adding auxiliary state to store the partial amount of computatin.

% --------------------------------------------------------------------------------
\subsubsection{Temporal tiling}

In most computations there is a special axis that represents time and
moves only from past to future. The data along other axes represent
(potentially independent) features.

This is an easy requirement to enforce if the computation has no memory,
e.g., in the following example of Python code using Pandas

\begin{verbatim}
df1 =
                      a         b
2023-01-01 08:00:00   10        20
2023-01-01 08:30:00   10        20
2023-01-01 09:00:00   10        20
2023-01-01 09:30:00   10        20

df2 =
                      a         b
2023-01-01 08:00:00   10        20
2023-01-01 08:30:00   10        20
2023-01-01 09:00:00   10        20
2023-01-01 09:30:00   10        20
\end{verbatim}

\begin{lstlisting}[language=Python]
dfo = df1 + df2
\end{lstlisting}

DataFlow can partition df1 and df2 in different slices obtaining the
same result, e.g.,

\begin{verbatim}
df1_0 =
                      a         b
2023-01-01 08:00:00   10        20
2023-01-01 08:30:00   10        20

df2_0 =
                      a         b
2023-01-01 08:00:00   10        20
2023-01-01 08:30:00   10        20
\end{verbatim}

\begin{lstlisting}[language=Python]
dfo_0 = df1_0 + df2_0
\end{lstlisting}

\begin{verbatim}
df1_1 =
                      a         b
2023-01-01 09:00:00   10        20
2023-01-01 09:30:00   10        20

df2_1 =
                      a         b
2023-01-01 09:00:00   10        20
2023-01-01 09:30:00   10        20
\end{verbatim}

\begin{lstlisting}[language=Python]
dfo_1 = df1_1 + df2_1
dfo = pd.concat([dfo_1, dfo_2])
\end{lstlisting}

Consider the case of a computation that relies on past values

\begin{lstlisting}[language=Python]
dfo = df1.diff()
\end{lstlisting}

This computation to be invariant to slicing needs to be fed with a
certain amount of previous data

\begin{verbatim}
df1_0 =
                      a         b
2023-01-01 08:00:00   10        20
2023-01-01 08:30:00   10        20
\end{verbatim}

\begin{lstlisting}[language=Python]
dfo_0 = df1_0.diff()
dfo_0 = dfo_0["2023-01-01 08:00:00":"2023-01-01 08:30:00"]
\end{lstlisting}

\begin{verbatim}
df1_1 =
                      a         b
2023-01-01 08:30:00   10        20
2023-01-01 09:00:00   10        20
2023-01-01 09:30:00   10        20
\end{verbatim}

\begin{lstlisting}[language=Python]
dfo_1 = df1_1.diff()
dfo_1 = dfo_1["2023-01-01 09:00:00":"2023-01-01 09:30:00"]
dfo = pd.concat([dfo_1, dfo_2])
\end{lstlisting}

In general as long as the computation doesn't have infinite memory
(e.g., an exponentially weighted moving average)

This is possible by giving each nodes data that has enough history

Many interesting computations with infinite memory (e.g., EMA) can also
be decomposed in tiles with using some algebraic manipulations -
TODO(gp): Add an example of EMA expressed in terms of previous

Given the finite nature of real-world computing (e.g., in terms of
finite approximation of real numbers, and bounded computation) any
infinite memory computation is approximated to a finite memory one. Thus
the tiling approach described above is general, within any desired level
of approximation.

The amount of history is function of a node

% --------------------------------------------------------------------------------
\subsubsection{Cross-sectional tiling}

\begin{itemize}
  \item The same principle can be applied to tiling computation
        cross-sectionally
  \item Computation that needs part of a cross-section need to be tiled
        properly to be correct
  \item TODO(gp): Make an example
\end{itemize}

% --------------------------------------------------------------------------------
\subsubsection{Temporal and cross-sectional tiling}

\begin{itemize}
  \item These two styles of tiling can be composed
  \item The tiling doesn't even have to be regular, as long as the constraints
        for a correct computation are correct
\end{itemize}

% --------------------------------------------------------------------------------
\subsubsection{Detecting incorrect tiled computations}

\begin{itemize}
  \item One can use the tiling invariance of a computation to verify that it
        is correct
  \item E.g., if computing a DAG gives different results for different tiled,
        then the amount of history to each node is not correct
\end{itemize}

% --------------------------------------------------------------------------------
\subsubsection{Benefits of tiled computation}

\begin{itemize}
  \item Another benefit of tiled computation is that future peeking (i.e., a
        fault in a computation that requires data not yet available at the
        computation time) can be detected by streaming the data with the same
        timing as the real-time data would do
\end{itemize}

A benefit of the tiling is that the compute frame can apply any tile
safe transformation without altering the computation, e.g.,

\begin{itemize}
  \item vectorization across data frames
  \item coalescing of compute nodes
  \item coalescing or splitting of tiles
  \item parallelization of tiles and nodes across different CPUs
  \item select the size of a tile so that the computation fits in memory
\end{itemize}

% --------------------------------------------------------------------------------
\subsubsection{Batch vs streaming}

Once a computation can be tiled, the same computation can be performed
in batch mode (e.g., the entire data set is processed at once) or in
streaming mode (e.g., the data is presented to the DAG as it becomes
available) yielding the same result

This allows a system to be designed only once and be run in batch (fast)
and real-time (accurate timing but slow) mode without any change

In general the more data is fed to the system at once, the more likely
is to being able to increase performance through parallelization and
vectorization, and reducing the overhead of the simulation kernel (e.g.,
assembling/splitting data tiles, bookkeeping), at the cost of a larger
footprint for the working memory

In general the smaller the chunks of data are fed to the system (with
the extreme condition of feeding data with the same timing as in a
real-time set-up), the more unlikely is a fault in the design it is
(e.g., future peeking, incorrect history amount for a node)

Between these two extremes is also possible to chunk the data at
different resolutions, e.g., feeding one day worth of data at the time,
striking different balances between speed, memory consumption, and
guarantee of correctness

% ================================================================================
\subsection{Vectorization}

% --------------------------------------------------------------------------------
\subsubsection{Vectorization}
Vectorization is a technique for enhancing the performance of computations by
simultaneously processing multiple data elements with a single instruction,
leveraging the capabilities of modern processors (e.g., SIMD (Single
Instruction, Multiple Data) units).

% --------------------------------------------------------------------------------
\subsubsection{Vectorization in KaizenFlow}
Given the DataFlow format, where features are organized in a hierarchical
structure, KaizenFlow allows one to apply an operation to be applied across the
cross-section of a dataframe.
In this way KaizenFlow exploits Pandas and NumPy data manipulation and
numerical computing capabilities, which are in turns built on top of low-level
libraries written in languages like C and Fortran. These languages provide
efficient implementations of vectorized operations, thus bypassing the slower
execution speed of Python loops.

% --------------------------------------------------------------------------------
\subsubsection{Example of vectorized node in KaizenFlow}

TODO

% ================================================================================
\subsection{Incremental, cached, and parallel execution}

% --------------------------------------------------------------------------------
\subsubsection{DataFlow and functional programming}

The DataFlow computation model shares many similarity with functional
programming:

\begin{itemize}
  \item Data immutability: data in dataframe columns is typically added or
        replaced. A node in a DataFlow graph cannot alter data in the nodes
        earlier in the graph.
  \item Pure functions: the output of a node depends only on its input values
        and it does not cause observable side effects, such as modifying a global
        state or changing the value of its inputs
  \item Lack of global state: nodes do not rely on data outside their scope,
        especially global state
\end{itemize}

% --------------------------------------------------------------------------------
\subsubsection{Incremental computation}

Only parts of a compute graph that see a change of inputs need to be recomputed.

Incremental computation is an approach where the result of a computation is
updated in response to changes in its inputs, rather than recalculating
everything from scratch

% --------------------------------------------------------------------------------
\subsubsection{Caching}

Because of the "functional" style (no side effects) of data flow, the output of
a node is determinstic and function only of its inputs and code.

Thus the computation can be cached across runs. E.g., if many DAG simulations
share the first part of simulation, then that part will be automatically cached
and reused, without needing to be recomputed multiple times.

TODO(gp): Explain the algo in more detail.
TODO(gp): Add a picture.

% --------------------------------------------------------------------------------
\subsubsection{Parallel execution}

Parallel and distributed execution in KaizenFlow is supported at two different
levels:
\begin{itemize}
  \item Across runs:
        given a list of \verb|Config|, each describing a different system, each
        simulation can be in parallel because they are completely independent.
  \item Intra runs:
        each DataFlow graph can be run exploiting the fact that nodes
\end{itemize}

In the current implementation for intra-run parallelism Kaizen flow relies on
\verb|Dask|
For across-run parallelism KaizenFlow relies on \verb|joblib| or \verb|Dask|

Dask extends the capabilities of the Python ecosystem by providing an efficient
way to perform parallel and distributed computing.

Dask supports various forms of parallelism, including multi-threading,
multi-processing, and distributed computing. This allows it to leverage
multiple cores and machines for computation.

When working in a distributed environment, Dask distributes data and
computation across multiple nodes in a cluster, managing communication and
synchronization between nodes. It also provides resilience by re-computing lost
data if a node fails.

% ================================================================================
\subsection{Train and predict}
% from all.train_and_predict_phases.explanation.md

%- Nodes can be stateless or statefull
%- Load/save the state
%- Different types of cross-validation, rolling, in-sample, in-sample/oos

% --------------------------------------------------------------------------------
\subsubsection{Stateful nodes}

A DAG node is stateful if it uses data to learn parameters (e.g., linear
regression coefficients, weights in a neural network, support vectors in a SVM)
during the \verb|fit| stage, that are then used in a successive
\verb|predict| stage.

The state is stored inside the implementation of the node.

The state of stateful DAG node varies during a single simulation.

TODO: Add snippet of code showing stateful node.

% --------------------------------------------------------------------------------
\subsubsection{Stateless nodes}
A DAG node is stateless if the output is not dependent on previous \verb|fit|
stages. In other words the output of the node is only function of the current
inputs and of the node code, but not from inputs from previous tiles of inputs.

A stateless DAG node emits the same output independently from the current and
previous \verb|fit| vs \verb|predict| phases.

A stateless DAG node has no state that needs to be stored across a simulation.

% --------------------------------------------------------------------------------
\subsubsection{Loading and saving node state}

Each stateful node needs to allow saving and loading its state on demand of the
framework.

A stateless node should emit an empty state when saving and assert in case a
non-empty state is presented during a load phase.

KaizenFlow simulation kernel orchestrate loading and saving nodes for an entire
DAG to serialize and deserialize a DAG to disk.

TODO(gp): Add an example of data layout

KaizenFlow allows one to load the state of a DAG for further analysis. E.g., one
might want to see how weights of a linear model evolve over time in a rolling
window simulation.

% --------------------------------------------------------------------------------
\subsubsection{Different types of learning}

TODO(gp): Improve.

Cross-validation is a statistical method used to estimate the skill of machine
learning models. It is primarily used to assess how the results of a
statistical analysis will generalize to an independent data set. In the context
of time series data, where the temporal order of data points is important,
special forms of cross-validation, like in-sample and rolling-window
cross-validation, are used. Here's a brief explanation of these methods:

In-Sample Cross-Validation:

In this method, the entire dataset is used for both training and testing. The
model is trained on a certain portion of the data (the training set) and then
tested on the remaining data (the test set). One common approach is to split
the data chronologically. For example, the model might be trained on the first
80% of the dataset (in chronological order) and tested on the remaining 20%.
In-sample cross-validation is useful for time series data because it respects
the chronological order of observations. Rolling-Window (or Walk-Forward)
Cross-Validation:

This method is more sophisticated and particularly suited for time series data.
In rolling-window cross-validation, the model is trained on a fixed-size window
of data and then makes predictions for the subsequent time period.
After each training and testing phase, the window is "rolled" forward, which
means that the model is retrained on a new window of data including the most
recent observations.
For example, if you have monthly data for 10 years, you might train the model
on the first year of data and test it on the next month. Then, you roll the
window forward by one month (so the training data now starts from month 2 and
goes up to month 13) and test on the 14th month. This process continues until
you have tested the model on all available data.
This method is especially useful for evaluating the model's performance over
time and for datasets where the relationship between input and output variables
changes.
Both methods have their advantages and are chosen based on the specific
characteristics of the dataset and the research or business problem. In-sample
cross-validation is simpler and can be a good choice when the dataset is not
very large, while rolling-window cross-validation is more robust for evaluating
time-series models as it mimics the real-world scenario of training a model on
past data and predicting future events.

% ================================================================================
\subsection{Observability and debuggability}

% --------------------------------------------------------------------------------
\subsubsection{Running a DAG partially}
KaizenFlow allows one to run nodes and DAGs in a notebook during design, analysis,
and debugging phases, and in a Python script during simulation and production
phases.

It is possible to run a DAG up to a certain node to iterate on its design and debug.

TODO: Add example

% --------------------------------------------------------------------------------
\subsubsection{Replaying a DAG}

Each DAG node can:

\begin{itemize}
  \item capture the stream of data presented to it during either a simulation
        and real-time execution
  \item serialize the inputs and the outputs, together with the knowledge
        timestamps
  \item play back the outputs
\end{itemize}

KaizenFlow allows one to describe a cut in a DAG and capture the inputs and
outputs at that interface. In this way it is possible to debug a DAG
replacing all the components before a given cut with a synthetic one
replaying the observed behavior together with the exact timing in terms
of knowledge timestamps.

This allows one to easily:

\begin{itemize}
  \item capture failures in production and replay them in simulation for
        debugging
  \item write unit tests using observed data traces
\end{itemize}

KaizenFlow allows each node to automatically save all the inputs and
outputs to disk to allow replay and analysis of the behavior with high
fidelity.

% ================================================================================
\subsection{Profiling DataFlow execution}
% TODO(Grisha)

%- Information about each node execution time, memory footprint, basic stats
%about inputs and outputs
%- Notebooks that allow you to compute some stats
%- Maybe add a box plot to give an idea

% ================================================================================
\subsection{DataFlow and the Python data stack}
%- Describe how other libraries can be integrated in DataFlow
%- E.g., sklearn, gluonts, statsmodel, Pandas, numpy
%
%- Show how a sklearn node can be used

% ###############################################################################
\section{ML Ops}

% ###############################################################################
\section{Application of KaizenFlow to quant finance}

% Specifics of how we use KaizenFlow for finance

% ###############################################################################
\section{Comparison to other computing framework}

% ================================================================================
\subsection{Comparison principles}

% --------------------------------------------------------------------------------
\subsubsection{Static vs Dynamic}
Static Nature: TensorFlow uses a static computational graph, meaning the graph
is defined before it is run.

Dynamic Nature: Unlike some other frameworks that use static computational
graphs, PyTorch operates on a dynamic (or "eager") computational graph. This
means the graph is built on-the-fly as operations are executed. This property
is known as the "define-by-run" paradigm.

% --------------------------------------------------------------------------------
\subsubsection{Fundamental data structure}
Tensors: The fundamental data structure in PyTorch is the Tensor, which is
similar to NumPy arrays but with additional capabilities to operate on GPUs for
accelerated computing.

Events

Dataframes

% --------------------------------------------------------------------------------
\subsubsection{Unified support for batch and streaming}
This allows developers to build data processing pipelines that can handle both
batch and stream data processing in a unified manner. This means you can write
your data processing logic once and run it on different processing engines
without major code changes.

% --------------------------------------------------------------------------------
\subsubsection{Python support}
Integration with NumPy, Pandas

Users can extend its functionalities using Python’s features, and it allows for
the integration of other Python libraries.

% --------------------------------------------------------------------------------
\subsubsection{Native time series support}
Knowledge time

% --------------------------------------------------------------------------------
\subsubsection{Processing engine agnostic}
Provides a portable API that can run on
multiple processing engines, making it vendor-agnostic. This portability allows
organizations to switch between processing engines without rewriting their data
processing code.

% --------------------------------------------------------------------------------
\subsubsection{Windowing and event time processing}
supports windowing and event time processing for handling data that arrives out
of order in stream processing scenarios. This is crucial for real-time
analytics and aggregations.

% --------------------------------------------------------------------------------
\subsubsection{Support for dense computation}

% --------------------------------------------------------------------------------
\subsubsection{Scalability and fault tolerance}
Scalability: designed to scale horizontally, allowing it to handle
large volumes of data by distributing the processing across multiple machines
or clusters.

Fault Tolerance: The framework provides built-in mechanisms for handling
failures and ensuring data correctness during processing.

% ================================================================================
\subsection{Pytorch}
PyTorch is an open-source machine learning library. It is widely used for deep
learning applications and is known for its ease of use, flexibility, and
dynamic computational graph.

Fundamental data structure: tensors

% ================================================================================
\subsection{TensorFlow}
Fundamental data structure: tensors

% ================================================================================
\subsection{Dask}
Fundamental data structure: dataframe, array, sets

% ================================================================================
\subsection{Apache Spark}

See \cite{ZhChFrShSt10}, \cite{ShMoAlNa19}. The motivations behind Spark are
different from ours. In particular, efficiently handling MapReduce-like
workloads (see \cite{DeGh08}) while reusing distributed data sets were key
design concerns.

% ================================================================================
\subsection{Apache Flink}

% ================================================================================
\subsection{Apache Beam}

Apache Beam is an open-source, unified batch and stream processing framework
that provides a way to create data processing pipelines. It was originally
developed by Google as the Dataflow Model and later open-sourced as Apache
Beam.

- Static vs Dynamic: ?
- Fundamental data structure: events
- Unified support for batch and streaming: yes
- Python support: Yes
- Python data stack support: No
- Native time series support: No
- Processing engine agnostic: Yes
- Windowing and event time processing: Yes
- Support for dense computation: No
- Scalability and fault tolerance: Yes

See \cite{Aketal13}, \cite{Aketal15}.

% ================================================================================
\subsection{Apache Storm}

Apache Storm is an open-source, real-time stream processing framework designed
for processing and analyzing continuous data streams. It was originally
developed by Twitter and later open-sourced as part of the Apache Software
Foundation.

% ================================================================================
\subsection{Spark Streaming}

% ================================================================================
\subsection{Apache Samza}

% ================================================================================
\subsection{Kafka Streams}

% ================================================================================
\subsection{Azure Stream Analytics}

% ###############################################################################
\bibliography{KaizenFlowBib}
\bibliographystyle{amsplain}

\begin{thebibliography}{10}

  \bib{Kleppmann17}{book}{
    title={Designing data-intensive applications: The big ideas behind reliable, scalable, and maintainable systems},
    author={Kleppmann, Martin},
    year={2017},
    publisher={" O'Reilly Media, Inc."}
  }


  \bib{Aketal13}{article}{
    author={Akidau, Tyler},
    author={Balikov, Alex},
    author={Bekiro\u{g}lu, Kaya},
    author={Chernyak, Slava},
    author={Haberman, Josh},
    author={Lax, Reuven},
    author={McVeety, Sam},
    author={Mills, Daniel},
    author={Nordstrom, Paul},
    author={Whittle, Sam},
    title={MillWheel: Fault-Tolerant Stream Processing at Internet Scale},
    year={2013},
    publisher={VLDB Endowment},
    volume={6},
    number={11},
    url={https://doi.org/10.14778/2536222.2536229},
    doi={10.14778/2536222.2536229},
    month={aug},
    pages={1033-1044},
    numpages={12},
  }

  \bib{Aketal15}{article}{
  author={Akidau, Tyler},
  author={Bradshaw, Robert},
  author={Chambers, Craig},
  author={Chernyak, Slava},
  author={Fern\'{a}ndez-Moctezuma, Rafael J.},
  author={Lax, Reuven},
  author={McVeety, Sam},
  author={Mills, Daniel},
  author={Perry, Frances},
  author={Schmidt, Eric},
  author={Whittle, Sam},
  title={The Dataflow Model: A Practical Approach to Balancing Correctness, Latency, and Cost in Massive-Scale, Unbounded, out-of-Order Data Processing},
  year={2015},
  publisher={VLDB Endowment},
  volume={8},
  number={12},
  doi={10.14778/2824032.2824076},
  month={Aug},
  pages={1792-1803},
  }

  \bib{CaEwHaKaMaTz15}{article}{
    author={Carbone, Paris},
    author={Ewen, Stephan},
    author={Haridi, Seif},
    author={Katsifodimos, Asterios},
    author={Markl, Volker},
    author={Tzoumas, Kostas},
    title={Apache Flink: Stream and Batch Processing in a Single Engine},
    year={2015},
    journal={IEEE Data Engineering Bulletin},
    volume={38},
    month={1},
  }

  \bib{DeGh08}{article}{
    author={Dean, Jeffrey},
    author={Ghemawat, Sanjay},
    title={MapReduce: Simplified Data Processing on Large Clusters},
    year={2008},
    address={New York, NY, USA},
    volume={51},
    number={1},
    url={https://doi.org/10.1145/1327452.1327492},
    doi={10.1145/1327452.1327492},
    publisher={Association for Computing Machinery},
    journal={Commun. ACM},
    month={jan},
    pages={107–113},
  }

  \bib{Mc10}{article}{
  author={McKinney, Wes},
  title={Data Structures for Statistical Computing in Python},
  booktitle={Proceedings of the 9th Python in Science Conference},
  pages={56-61},
  year={2010},
  editor={St\'efan van der Walt and Jarrod Millman}
  doi={10.25080/Majora-92bf1922-00a}
  }

  \bib{ShMoAlNa19}{article}{
    author={Shaikh, Eman},
    author={Mohiuddin, Iman},
    author={Alufaisan, Yasmeen},
    author={Nahvi, Irum},
    title={Apache Spark: A Big Data Processing Engine},
    year={2019},
    pages={1-6},
    doi={10.1109/MENACOMM46666.2019.8988541},
  }

  \bib{ZhChFrShSt10}{article}{
    author={Zaharia, Matei},
    author={Chowdhury, Mosharaf},
    author={Franklin, Michael J.},
    author={Shenker, Scott},
    author={Stoica, Ion},
    title={Spark: Cluster Computing with Working Sets},
    year={2010},
    publisher={USENIX Association},
    address={USA},
    pages={10},
    doi={10.5555/1863103.1863113}
  }

  % J. Dean and S. Ghemawat. MapReduce: Simplified Data Processing on Large Clusters. In Proc. of the Sixth Symposium on Operating System Design and Implementation (OSDI), 2004.
  %[1] D. J. Abadi et al. Aurora: A New Model and Architecture for Data Stream Management. The VLDB Journal, 12(2):120–139, Aug. 2003.
  %[2] T. Akidau et al. MillWheel: Fault-Tolerant Stream Processing at Internet Scale. In Proc. of the 39th Int. Conf. on Very Large Data Bases (VLDB), 2013.
  %[3] A. Alexandrov et al. The Stratosphere Platform for Big Data Analytics. The VLDB Journal, 23(6):939–964, 2014.
  %[4] Apache. Apache Hadoop. http://hadoop.apache.org, 2012.
  %[5] Apache. Apache Storm. http://storm.apache.org, 2013.
  %[6] Apache. Apache Flink. http://flink.apache.org/, 2014.
  %[7] Apache. Apache Samza. http://samza.apache.org, 2014.

  %    \bib{}{article}{
  %      author={,},
  %      title={},
  %      date={},
  %    }

\end{thebibliography}

\end{document}
