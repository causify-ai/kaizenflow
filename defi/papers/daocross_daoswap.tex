\documentclass[11pt, reqno]{amsart}
\usepackage{amsfonts, amssymb, amscd, amsrefs}
\usepackage{graphicx}
\usepackage{hyperref}
\usepackage{slashed}
\usepackage{fullpage}
% Prevent from repositioning tables.
\usepackage{float}

% TODO(gp): It aborts with
% ! Undefined control sequence.
%<argument> ...on\endcsname \protect \@secnumpunct
%
%l.164 \subsection{Breaking the symmetry}
% https://tex.stackexchange.com/questions/165930/bold-and-italic-subsection-title-with-custom-font-size
%\usepackage{titlesec}
%\titleformat{\section}
%{\normalfont\fontfamily{phv}\fontsize{12}{17}\bfseries}{\thesection}{1em}{}
%\titleformat{\subsection}
%{\normalfont\fontfamily{phv}\fontsize{12}{17}\bfseries\itshape}{\thesubsection}{1em}{}
%\titleformat{\subsection}
%  {\normalfont\fontsize{12}{17}\sffamily\bfseries\slshape}
%  {\thesubsection}
%  {1em}
%  {}

%\usepackage{titlesec}

%\titleformat*{\section}{\LARGE\bfseries}
%\titleformat*{\subsection}{\Large\bfseries}
%\titleformat*{\subsubsection}{\large\bfseries}
%\titleformat*{\paragraph}{\large\bfseries}
%\titleformat*{\subparagraph}{\large\bfseries}

% https://ctan.math.washington.edu/tex-archive/macros/latex/required/amscls/doc/amsthdoc.pdf
\newtheorem{thm}{Theorem}
\theoremstyle{definition}
\newtheorem{defn}{Definition}[subsection]
\newtheorem{problem}{Problem}[subsection]
\theoremstyle{remark}
\newtheorem{exmp}{Example}[subsection]
\newtheorem{rmk}{Remark}[subsection]

\newcommand{\bidbtc}{\mathrm{bid}_\mathrm{BTC}}
\newcommand{\askbtc}{\mathrm{ask}_\mathrm{BTC}}
\newcommand{\bideth}{\mathrm{bid}_\mathrm{ETH}}
\newcommand{\asketh}{\mathrm{ask}_\mathrm{ETH}}

\newcommand{\bidbase}{\mathrm{bid}_\mathrm{quote\;token}}
\newcommand{\askbase}{\mathrm{ask}_\mathrm{quote\;token}}
\newcommand{\bidquote}{\mathrm{bid}_\mathrm{base\;token}}
\newcommand{\askquote}{\mathrm{ask}_\mathrm{base\;token}}
% Use teletype for tokens (\texttt{...}), but do not allow italics
%  (\textnormal{...}).
\newcommand{\BTC}{\textnormal{\texttt{wBTC}}}
\newcommand{\ETH}{\textnormal{\texttt{ETH}}}
\newcommand{\DAI}{\textnormal{\texttt{DAI}}}
\newcommand{\USDC}{\textnormal{\texttt{USDC}}}
\newcommand{\USDT}{\textnormal{\texttt{USDT}}}
\newcommand{\tA}{\textnormal{\texttt{A}}}
\newcommand{\tB}{\textnormal{\texttt{B}}}
% Use non-italic teletype for order attributes in order notation.
\newcommand{\timestamp}{\textnormal{\texttt{timestamp}}}
\newcommand{\action}{\textnormal{\texttt{action}}}
\newcommand{\quantity}{\textnormal{\texttt{quantity}}}
\newcommand{\basetoken}{\textnormal{\texttt{base\_token}}}
\newcommand{\limitprice}{\textnormal{\texttt{limit\_price}}}
\newcommand{\quotetoken}{\textnormal{\texttt{quote\_token}}}
\newcommand{\depositaddress}{\textnormal{\texttt{deposit\_address}}}
%
\newcommand{\buy}{\textnormal{\texttt{buy}}}
\newcommand{\sell}{\textnormal{\texttt{sell}}}
\newcommand{\nan}{\textnormal{\texttt{nan}}}

\newcommand{\midpoint}{\mathrm{midpoint}}

% Use this if using `\contrib[]{...}`.
%\makeatletter\let\@wraptoccontribs\wraptoccontribs\makeatother

\begin{document}

\title{DELOC: A protocol for Decentralized Exchange Limit Order Crossing}

\author{Giacinto Paolo Saggese}
\author{Paul Smith}

\thanks{With contributions from
	Samarth KaPatel,
	Grisha Pomazkin,
	Juraj Smeriga,
	Danya Tikhomirov,
	Nina Trubacheva,
	Dan Yachmenev,
	Vladimir Yakovenko,
	and
	Tamara Zhordaniya}

\date{\today}

\maketitle

\tableofcontents


% ###############################################################################

\section{Introduction}

DELOC\footnote{Temporary code name}
is a new protocol that enables decentralized token exchange using limit
orders. It brings together the advantages of trading efficiency and ease-of-use
of centralized exchanges with the transparency and self-custody of
decentralized approaches, without the risk of impermanent loss, contrary to
automatic market maker-based approaches.

% ===============================================================================

\subsection{Token swap as financial primitive}
Swapping tokens is one of the most important and widely used primitives in
decentralized finance.
In March 2023, the average daily trading volume in cryptocurrencies was over
\$34 Billion USD, after reaching a peak of \$516 Billion USD per day in May 20,
2021\footnote{https://www.statista.com/statistics/1272903/cryptocurrency-trade-volume}.

% TODO(gp): We should compute this ourselves using spot, futures, etc

Trading volume on decentralized exchanges has been increasing over time and now
is approaching 20\% of the total traded cryptocurrency
volume\footnote{https://www.theblock.co/data/decentralized-finance/dex-non-custodial/dex-to-cex-spot-trade-volume}.
As of March 2023, Uniswap has transacted more than \$1.4 Trillion USD and 139
million trades\footnote{https://uniswap.org/}.
The move from centralized to decentralized exchanges has been accelerating due
to scandals involving CEXs (e.g., FTX bankruptcy), regulatory crackdown, and
thanks to end-users' increased embrace of the principles of self-custody and
decentralization.
We believe that this trend towards larger trading volumes and will only
accelerate in the future.

% ===============================================================================

\subsection{Limit orders as an interface to token swap}
A limit order expresses a commitment to buy or to sell an asset, up to
a certain quantity, at a price that is as good or better than the limit price.
As supply and demand are represented by quantity available at a given price,
limit orders provide a way for buyers and sellers to participate in price
discovery and exchange.

We adopt the perspective that limit orders may be thought of as an interface
to market participation. That is, limit orders provide participants with a
means to engage with markets. We further promote this interface as one that
is natural, flexible, and elegant. It is natural in that it expresses
supply or demand directly in terms of quantity and price, with a clearly
defined maximum possible exposure in a commitment to trade. The interface is
flexible because it allows participants to construct personalized supply and
demand curves by combining multiple limit orders. The elegance follows from the
simplicity and ease-of-use of the interface.

As consequence of these benefits, limit orders and their aggregation into
limit order books (LOBs) are ubiquitous in traditional financial markets
and dominate cryptocurrency trading on centralized exchanges. Interestingly,
however, limit orders currently do not play as prominent a role in the realm of
decentralized finance. Instead, alternative approaches that rely upon
automated market makers (AMMs) have taken center stage, and these approaches
offer a different set of advantages and disadvantages.

% ===============================================================================

\subsection{How exchanges create value}
To better understand the comparative advantages and disadvantages of different
approaches to token exchange, we step back and consider the purpose and
functions of exchanges.
Exchanges create value by bringing together different types of participants,
such as investors, traders, hedgers, brokers, arbitrageurs, and market makers.
Market participants have different goals (e.g., hedging, investing,
speculating), horizons (from long-term investment to low-latency trading),
risk tolerance, liquidity preferences, and beliefs about security values. The
opportunity to trade arises from these differences.
By providing a common meeting ground, exchanges facilitate matching supply
and demand.

Exchanges have several roles:
\begin{itemize}
	\item Establish the rules of the trading process and institutional roles, so
          that the trading process is structured, monitored, and standardized
	\item Provide a certain amount of oversight by monitoring and certifying
          financial statements and governance procedures
    \item Generate market data for market participants, such as trades and
          quote changes
\end{itemize}

Exchanges are compensated for this value creation by collecting transaction
fees as a small percent of traded volume.

Typically, an exchange requires two components:
\begin{enumerate}
	\item Trade matching: participants find a counterparty agreeing on quantity
	      and price of the assets to swap (this includes price discovery)
    \item Trade settlement (clearing): the assets are actually swapped after
          finding matching counterparties
\end{enumerate}

In decentralized finance, a token swap can be accomplished in different ways,
e.g., using
\begin{itemize}
	\item a centralized exchange (CEX) (e.g., Coinbase, Binance, OkX)
	\item a decentralized exchange (DEX) (e.g., Uniswap, SushiSwap)
\end{itemize}

We consider CEXs a temporary bridge between traditional finance and the
new vision of decentralized finance and we do not consider them a viable
long-term solution to the problems that DeFi is poised to address.

Centralized exchanges are usually organized around different versions of a
central limit order book (CLOB), whereas decentralized exchanges have been
organized as
\begin{itemize}
	\item off-chain order books (e.g., 1inch)
	\item on-chain order books (e.g., SwapSwap, KyberSwap)
	\item automated market makers (AMM) (e.g., Uniswap)
\end{itemize}

Off-chain solutions can run into custodial and censorship issues that stand
in direct conflict with the ethos of decentralization and self-determination.
Unfortunately, limitations of current blockchain technology restrict
the amount of computation that can be performed on-chain, although these limits
are continously being removed by advancements in research and blockchain
technology (e.g., layer 2 chains, optimistic and zero-knowledge rollups).
Thus we consider solutions based on self-custody that do not require
trusting a third party (or at least allow one to verify its fairness) in line
with the principles of decentralization.

Off-chain order books are an hybrid version of CEX and DEX where the price
discovery and trade matching happens off-chain, while the trade settlement is
performed on-chain.

On-chain order book matching has the advantages of being custodial, since users
are completely in charge of their funds.
A major disadvantage of using on-chain limit order books in decentralized
exchange is cost of computation. Continual submission and cancelation of orders
incurs costs, as does the ongoing process of matching supply and demand.

% ===============================================================================

\subsection{The advantages of DELOC}

DELOC retains the key advantages of on-chain decentralized exchange while
overcoming the issues of cost. Additionally, it includes an implementation
that preserves the familiar LOB interface but deepens the pool of available
liquidity in matching supply and demand.

DELOC addresses cost issues primarily in two ways: (1) discretizing time; and
(2) scaling with layer 2 solutions. By addressing these issues, we combine the
novel advantages of decentralized exchange with the utility and ease-of-use of
the familiar interface of limit orders.

Behind the interface, we propose two smart
contracts\footnote{https://ethereum.org/en/developers/docs/smart-contracts/}
for
ERC-20\footnote{https://ethereum.org/en/developers/docs/standards/tokens/erc-20/}
token exchange that match supply and demand differently depending upon whether
there exists an external reference price (DaoCross) or whether the contract
also performs the role of price discovery (DaoSwap).

DaoCross is a decentralized liquidity pool that crosses buy and sell orders
with respect to an external reference price, i.e., a price oracle. This allows
traders to interact directly with each other and share price improvement with
respect to exchanges, e.g., by trading at bid-ask midpoint. Additionally, order
intentions are not publicly disclosed prior to a cross, which enables block or
other high-volume traders to execute quickly with minimal market impact.

DaoSwap performs price discovery by matching supply and demand at regular time
intervals. In the simplest case, the clearing price is determined by a
Walrasian-style auction (\cite{Wa}). In cases where liquidity is pooled more
broadly, the auction clearing mechanism satisfies a collection of conditional
inequalities.

While DaoSwap and DaoCross differ in how the token exchange rate is determined,
both share several common advantages:
\begin{enumerate}
	\item Low cost:
	      they implement a DeFi primitive that allows trading tokens
          peer-to-peer without incurring spread costs
	\item Capital efficiency:
	      participants provide liquidity only when
	      they wish to trade, and only in the amount they wish to trade
	\item No impermanent loss:
	      liquidity providers are not exposed to the risk of impermanent loss
	      (a form of unrealized loss that becomes realized upon withdrawing
	      liquidity), unlike the case with other decentralized exchange protocols
	      based on automated market makers
	\item No intermediary custody:
	      exchanged tokens always remain under the control of their respective
	      owners
	\item No rent extraction from high-frequency traders:
	      speed alone does not confer an advantage to traders, as the discrete
	      timing prevents predatory tactics such as front-running, limit order
	      scalping, and spoofing, which are known issues seen with continuous
	      limit order books
	\item Ease-of-use: the interaction between buyers and sellers occurs
	      through smart contracts, with a website front-end available
\end{enumerate}

% ###############################################################################

\section{Matching liquidity}
Suppose there are two parties, Alice an $\ETH$ holder and Bob a $\BTC$ holder
(wrapped Bitcoin, an ERC-20 compliant coin that tracks Bitcoin), who wish to
swap tokens at a competitive rate. Suppose also that there are many
additional $\ETH$ and $\BTC$ holders who may be interested in participating in a
swap. How should the liquidity be pooled? At first glance, it appears that
there should be a single pool of liquidity. The wrinkle manifests in determining
how to express a desire to trade. A simple expression of a desire to trade is
a limit order representing a commitment to trade up to $q$ in quantity of
a token at a token exchange rate up to $p$. But what if some parties express
quantity in terms $\ETH$ and some in terms of $\BTC$, and some parties
express limit prices in terms of $\ETH$ per $\BTC$, while others express limit
prices in terms of $\BTC$ per $\ETH$? Under these conditions, setting a single
clearing price and determining fill prioritization rules raises several
questions. We will consider and address these in the sequel, but to begin, we
will discuss a more constrained setting.

% ===============================================================================

\subsection{Breaking the symmetry}
To simplify, we follow the practice of FX (foreign exchange) futures markets,
which break the symmetry by specifying two non-interchangeable roles:
\begin{enumerate}
	\item a \emph{base} currency (for quantity)
	\item a \emph{quote} currency (for price)
\end{enumerate}
The roles are expressed by writing \emph{base currency/quote currency}, e.g.,
``EUR/USD".

A limit order then consists of a quantity expressed in the units of the base
currency and a price expressed in terms of the quote currency (per unit or
suitable multiple of base currency). See, for example, \cite{Cme23} (e.g.,
footnotes on page 25) for usage of this terminology. See \cite{CmeFx} for how
this works in practice for Euro FX contracts, where contract units are
denominated in Euros and price is quoted in U.S. dollars and cents per Euro
increment.

By breaking the symmetry, one may run a standard limit order book, hold a
classical Walrasian-style auction, and handle size quantization effects by
using one of a variety of priority rules based on quantity, price, and time.
See \cite{BeLaLiVa22} for discussion around priority rule variations and their
effects on market quality outcomes.

An effect of breaking the symmetry in this way is that either liquidity for a
pair of currencies may be expressed in only one of the two forms, or liquidity
for a pair of currencies is split across two separate contracts (with roles of
base and quote token reversed), each with its own limit order book.
Using our example, this would mean treating $\BTC/\ETH$ and $\ETH/\BTC$ as
separate token pairs, even though the union of the underlying parties and
sources of liquidity are typically the same.

% ===============================================================================

\subsection{Basket case}
It seems plausible that by retaining the symmetry in the two-token case and
instead treating a pair of tokens as a single source of liquidity, one may
contrive a more efficient exchange mechanism.

This minor expansion, however, suggests widening even further the universe of
eligible swaps. For example, if there are three tokens available for exchange,
one could contemplate many-for-one, one-for-many, or many-for-many token swaps,
all to be carried out in an atomic fashion, and perhaps with complex
constraints. A toy case along these lines one may consider is as follows:
\begin{problem}[Triangular liquidity]
Suppose there are the following three parties:
\begin{itemize}
	\item Party A, who holds $\BTC$ and wants $\ETH$
	\item Party B, who holds $\ETH$ and wants $\DAI$
	\item Party C, who holds $\DAI$ and wants $\BTC$
\end{itemize}
How should an auction be structured to ``best" facilitate exchange?
\end{problem}
While there may be other use cases and even demand for such auctions,
unfortunately, the problem in general is NP-complete (e.g., \cite{XiStWh05}),
which is a significant limiting factor in terms of feasibility and
auction expense.

In the sequel we propose such an exchange mechanism valid for any collection
of standard limit orders (all base/quote token limit orders may be pooled and
cleared in a single optimization).

% ###############################################################################

\section{Limit orders}
In the previous section, we discussed the notions of base currency and
quote currency from foreign exchange. Here we extend them to tokens.

% ===============================================================================

\subsection{Base/quote tokens}
A token (ordered) pair consists of a \emph{base token} and a
\emph{quote token}. The shorthand notation for expressing the pair is
\emph{base token/quote token} (e.g., ``$\BTC/\ETH$"), and as such indicates the
respective roles of the tokens.

By convention, \emph{quantity} to exchange is expressed in terms of the base
token, and \emph{price} is a token exchange rate expressed in terms of the
quote token per unit of base token (or multiple thereof).

\begin{exmp}[base/quote tokens]
In the pair $\BTC/\ETH$, $\BTC$ is the base token and $\ETH$
is the quote token. Interest to trade is expressed in terms of orders
with $\BTC$ as base token and $\ETH$ as quote token, as follows:
\begin{itemize}
    \item A ``buy" order represents a commitment to purchase the base token
          $\BTC$ and pay with the quote token $\ETH$
    \item A ``sell" order represents a commitment to sell the base token
          $\BTC$ and receive the quote token $\ETH$
    \item Quantity $q$ is in terms of the base token (``buy/sell a certain
          amount of $\BTC$")
    \item Limit price is in terms of the quote token (``buy/sell $\BTC$ with a
          limit price of $p$ $\ETH$ per $\BTC$")
\end{itemize}
A party who places a ``buy" order must have the quote token ($\ETH$) in custody,
i.e., in its wallet. A party who places a ``sell" order must have the base
token ($\BTC$) in custody.
\end{exmp}

% ===============================================================================

\subsection{Exchange price properties}
If prices of base and quote tokens are expressed in terms of a common currency
(e.g., USD) then it holds that:
\[
	p_{quote\_per\_base} = \frac{p_{quote}}{p_{base}},
\]
where the numeraire plays the role of a quote token for both $p_{quote}$ and
$p_{base}$ and is implicit.

Note that
\[
	p_{quote\_per\_base} =
	\frac{1}{p_{base\_per\_quote}}.
\]

% ===============================================================================

% TODO(gp): Maybe add a source_address as way to identify the user.
\subsection{Order attributes and notation}
We introduce the following tuple notation for a general limit order
(valid for both DaoCross and DaoSwap).
\begin{defn}[Limit order]
A DaoCross or DaoSwap limit order is represented by a tuple of the form
\begin{align*}
( & \timestamp,       \\
  & \action,          \\
  & \quantity,        \\
  & \basetoken,       \\
  & \limitprice,      \\
  & \quotetoken,      \\
  & \depositaddress )
\end{align*}
\end{defn}

The quantities are arranged to make the order simple to read in natural
language:
``At timestamp $\timestamp$ create an order to $\action$ up to a number
$\quantity$ $\basetoken$ tokens for a limit price of $\limitprice$ with respect
to the token $\quotetoken$ and deposit the resulting tokens at
$\depositaddress$."

We have previously discussed the roles of \action, \quantity, \basetoken,
\limitprice, and \quotetoken.
The roles of $\timestamp$ include determining eligibility in a swap and can
extend to influencing order fill priority.
The $\depositaddress$ attribute mirrors standard functionality
available in traditional finance, e.g., in purchasing T-Bills on
TreasuryDirect\footnote{https://www.treasurydirect.gov}. This feature
facilitates account management through the use of multiple wallets. For
example, a miner may have a supply of $\BTC$ collected and held in one wallet
which it would like to systematically diversify into $\ETH$ and perhaps other
tokens. In specifying a deposit address, it may use a separate wallet to collect
incoming $\ETH$.

\begin{exmp}[Limit order notation]
The order
\begin{center}
(\textnormal{\texttt{1678660406}},
\buy,
\textnormal{\texttt{3.2}},
\ETH,
\textnormal{\texttt{4.0}},
\BTC,
\textnormal{\texttt{0xdeadc0de}})
\end{center}
corresponds to the natural language description:
``At timestamp Mon Mar 13 2023 02:33:25 GMT+0000, the user commits to buy up to
3.2 units of $\ETH$ in exchange for $\BTC$ up to a $\limitprice$ of 4.0
$\BTC$ per $\ETH$ with proceeds deposited at \textnormal{\texttt{0xdeadc0de}}".
\end{exmp}

\subsubsection{Extracting attributes from an order}
Given an order $o_i =$ \texttt{(\buy, q, \tA, p, \tB)} we indicate with:
\begin{itemize}
	\item $action(o_i)$ the desired action ($\buy$ or $\sell$)
	\item $q_{base}(o_i)$ the maximum desired quantity $q$
	\item $base(o_i)$ the base token $\tA$
	\item $p(o_i)$ the limit price $p$ (in terms of quote per base)
	\item $quote(o_i)$ the quote token $\tB$
\end{itemize}

\subsubsection{Order short notation}
In the sequel, when clear from the context we may:
\begin{itemize}
\item omit $\timestamp$ and $\depositaddress$ when not relevant to the
      discussion
\item omit the (infinite) $\limitprice$ for market orders
\end{itemize}

% ===============================================================================

\subsection{Order clearing quantity and exchange rate}
In DaoCross, orders are collected from users during a finite period of time,
after which tokens are redistributed among users according to their limit
orders.

Equilibrium prices for the tokens are determined based on the available limit
orders and a criteria designed to maximize the wellfare of the participants in
the swap. At the same time, orders are matched in (generically) many-to-many
relationships.

Swaps between base/quote token pairs occur at a single exchange rate, i.e., the
clearing (or equilibrium) price is the same for all executed orders. On the
other hand, quantity exchanged is specific to each order and each order's
constraint on quantity exchanged cannot be violated. In the same way, each order
can be executed with a non-null quantity only when the limit price is compatible
with the clearing exchange rate.

In the sequel we use an asterisk to denote clearing quantity and exchange
rate:
\begin{itemize}
	\item $q_{base}^*(o_i)$ denotes quantity of the base token exchanged in a
	      swap from order $o_i$
	\item $p_{quote\_per\_base}^*$ denotes exchange rate between the quote
	      and base tokens for all the swaps
	      % \item Same definitions hold for the quote token $q_{quote}^*$ and $p_{quote}^*$
\end{itemize}

% ===============================================================================

\subsection{Execution of orders}
Next we introduce some examples of market order and limit order behavior when
a clearing exchange rate has been set.
In particular, consider a swap for the tokens $\ETH$, $\BTC$ and assume that
the exchange rate between $\ETH$ and $\BTC$ is fixed at 0.2 (i.e.,
0.2 $\BTC$ can be exchanged for 1 $\ETH$ and vice versa).

\begin{exmp}[Market order notation and clearing]
The following market orders omit $\timestamp$, $\limitprice$, and
$\depositaddress$ in favor of emphasizing the amounts of token exchanged:
\begin{itemize}
    \item An order \texttt{(\buy, 1.0, $\ETH$, $\BTC$)} means buying 1
          $\ETH$ in exchange for 0.2 $\BTC$
    \item An order \texttt{(\sell, 1.0, $\ETH$, $\BTC$)} means selling 1 $\ETH$,
          receiving the corresponding amount of 0.2 $\BTC$
    \item An order \texttt{(\buy, 1.0, $\BTC$, $\ETH$)} means buying 1 $\BTC$,
          paying with 5 $\ETH$
    \item An order \texttt{(\sell, 1.0, $\BTC$, $\ETH$)} means selling 
          1 $\BTC$ in return for 5 $\ETH$
\end{itemize}
\end{exmp}
Next we consider the behavior of limit orders under the same prevailing exchange
rate and we assume that there is sufficient supply of tokens to fully fill the
orders.

\begin{exmp}[Executable buy limit order]
A limit order
\[
    \texttt{(\buy, 1.0, $\ETH$, 0.5, $\BTC$)}
\]
means
``buy up to 1 $\ETH$ in exchange for $\BTC$ at a rate up to 0.5 $\BTC$ per $\ETH$."
In this case, since the price of one $\ETH$ is equal to 0.2 $\BTC$, the order
can be executed at the prevailing market rate.
\end{exmp}

\begin{exmp}[Non-executable buy limit order]
On the other hand, a limit order
\[
    \texttt{(\buy, 1.0, $\ETH$, 0.1, $\BTC$)}
\]
requires that the rate of $\BTC$ per $\ETH$ be lower than the current market
rate, and so the limit price prevents a token swap from being carried out.
\end{exmp}

\begin{exmp}[Non-executable sell limit order]
A limit order
\[
    \texttt{(\sell, 1.0, $\ETH$, 0.5, $\BTC$)}
\]
means
``sell up to 1 unit of $\ETH$ in exchange for $\BTC$ at a rate down to 0.5 $\BTC$ per $\ETH$."
Since the current rate of $\BTC$ per $\ETH$ is 0.2, which is below the limit
price of 0.5, the order cannot be executed.
\end{exmp}

% ===============================================================================

\subsection{Limit order as a set of inequalities}
Any limit order as defined above can be translated into inequalities involving
quantity of exchanged tokens and token exchange rates. Multiple limit orders
can be converted into a system of inequalities, which collectively constrain
potential exchange outcomes.

\subsubsection{Inequalities for buy order}
An order of the form $o_i =$ \texttt{(\buy, q, \tA, p, \tB)} means
``buy $q$ units of token $\tA$ in exchange for token $\tB$ with a limit price
up to $p$ $\tB$ per unit of $\tA$",
and it corresponds to the following constraint on realized quantity exchanged
and clearing exchange rate:
\begin{equation*}
	(p_{\tB\_per\_\tA}^* \leq p(o_i)) \land
	(0 \le q_\tA^*(o_i) \leq q(o_i)) \lor
	(q_\tA^*(o_i) = 0)
\end{equation*}
The constraint on the quantity exchanged is conditioned on the corresponding
price's satisfying the desired limit price constraint: if the desired limit price
constraint is not met, the swap cannot be carried out and the exchanged quantity
is 0.

% TODO(gp): Personally although less Latex-fancy this renders better IMO.
%\subsubsection{Example of inequalities for buy order}
\begin{exmp}[Inequalities for buy order]
An order of the form $o_1 = \texttt{(\buy, 2, \tA, 3, \tB)}$ means
``buy 2 units of token $\tA$ in exchange for token $\tB$ with a limit price up
to 3 $\tB$ per unit of $\tA$",
and it corresponds to the following constraint on realized quantity exchanged
and clearing exchange rate:
\begin{equation*}
    (p_{\tB\_per\_\tA}^* \leq 3) \land
    (0 \le q_\tA^*(o_1) \leq 2) \lor
    (q_\tA^*(o_1) = 0)
\end{equation*}
\end{exmp}

\begin{exmp}[Inequality for buy market order]
A market order (i.e., without a limit price) has no constraint on exchange rate
and thus is reduced to
\[
    0 \le q_\tA^*(o_1) \le q(o_i)
\]
\end{exmp}

\subsubsection{Inequalities for sell order}
An order of the form $o_i = \texttt{(\sell, q, \tA, p, \tB)}$ means
``sell $q$ units of token $\tA$ in exchange for token $\tB$ with a limit price
of at least $p$ $\tB$ per unit of $\tA$",
and it corresponds to the following constraint on realized quantity exchanged
and clearing exchange rate:
\begin{equation*}
	(p_{\tB\_per\_\tA}^* \geq p(o_i) \land
	(0 \le q_\tA^*(o_i) \leq q(o_i)) \lor
	(q_\tA^*(o_i) = 0)
\end{equation*}

\begin{exmp}[Inequalities for sell order]
In the same way, an order like $o_2 = \texttt{(sell, 3, \tA, 2, \tB)}$ means
``sell 3 units of token $\tA$ in exchange for token $\tB$ with a limit price of
at least 2 $\tB$ per unit of $\tA$", which corresponds to
\begin{equation*}
    (p_{\tB\_per\_\tA}^*(o_2) \geq 2) \land
    (0 \le q_\tA^*(o_2) \leq 2) \lor
    (q_\tA^*(o_2) = 0)
\end{equation*}
\end{exmp}

% ===============================================================================

\subsection{Order normalization}
When the exchange rate between two tokens is known, it is possible to convert
buy/sell orders for both $\tA$ and $\tB$ tokens into buy/sell orders where all
orders have token $\tB$ as the base token, according to the transformation below.

Let
\[
	o_1 = (\buy, q, \tA, p_{\tB\_per\_\tA}, \tB).
\]
Suppose that $p^*_{\tB\_per\_tA}$ is fixed and known. Then the order
\[
	o_2 = (\sell, q^\prime, \tB, 1 / p_{\tB\_per\_\tA}, \tA)
\]
may be handled in order crossing in the same way that $o_1$ may be
provided that
\[
	q^\prime = q \cdot p^*_{\tB\_per\_\tA}
\]

% TODO(gp): flesh out the examples
%p_A / p_B = c <-> q_A = c * q_B <-> q_A units of A can be exchanged with (const * q_B) units of B
%E.g., ETH / wBTC = 0.5 <-> 1 ETH = 0.5 wBTC <-> 2 ETH = 1 wBTC
%So 3 ETH can be exchanged with 1.5 wBTC
%
%q^*(o_1) = (buy, q_A, A, lp_B, B) = "buy at most q_A units of A, where the price for A is at most lp_B"
%q^*(o_1) <= q_A
%p_A / p_B <= lp_B
%
%(sell, q_B, B, lp_A, A) = "sell at most q_B units of B, where the price for B is at least lp_A"
%q^*(o_2) <= q_B
%p_B / p_A >= lp_A
%
%Since p_A / p_B <= lp_B <-> p_B / p_A >= 1 / lp_B and q_A = c * q_B, the first order o_2 = (sell, q_B, B, lp_A, A) is equivalent to (buy, c * q_B, A, 1 / lp_B, B).

% \subsubsection{Example of order equivalence}
%Example of order equivalence. Assume that Alice wants to buy 5 wBTC with ETH. The exchange rate ETH / wBTC is 0.1 (or equivalently 1 ETH corresponds to 10 wBTC). With her order, Alice wants to receive 5 wBTC in exchange for 0.5 ETH. Alice would be satisfied also with an order of selling 0.5 ETH against wBTC.
%
%q_BTC < 4
%p_ETH / p_BTC < 3
%
%q_BTC = 0 if (p_ETH / p_BTC >= 3) else q_BTC < 4

% ###############################################################################

\section{Reference clearing prices and prioritization}
DaoCross relies on a \emph{price oracle} for determining the effective
token exchange rate in a cross. The price oracle may come from a lit
exchange, centralized or decentralized, or even on-chain automated market
makers such as Uniswap (\cite[\S 2.2]{AdZiRo20}).

The case of using an automated market maker as a price oracle is relatively
straightforward, provided there is an automated market maker trading the
target currency pair with sufficient liquidity.

To use an exchange as a price reference, we must consider some additional
steps. First, exchanges typically use a dollar stablecoin (e.g., $\USDC$,
$\USDT$) as the quote currency and all other currencies as base currencies. Our
example of $\BTC/\ETH$ would be considered a cross pair in the world of
traditional finance. Because most cross pairs are not traded directly on
exchanges, we must derive a clearing price from pairs involving stablecoins.

% ===============================================================================

\subsection{Lit exchange bid-ask midpoint reference price}
Let $\bidbtc, \askbtc$ be stablecoin-denominated top-of-book bid-ask prices
for $\BTC$ (e.g., expressed in $\USDC$), and $\bideth, \asketh$ be the analogous
prices for $\ETH$. Then, a commitment to buy one $\BTC$ and pay $\ETH$ would
clear at a midpoint price of
\[
	\BTC/\ETH_{\midpoint} = \frac{\bidbtc + \askbtc}{\bideth + \asketh}
\]
expressed in $\BTC$ per $\ETH$. Note that if the dollar price of BTH is significantly
more than the dollar price of ETH, then this price implies paying many
multiples of ETH for BTC (as is the case at the time of this writing).

In general, DaoCross may use the following reference price for a base/quote
token pair, where the bids and asks come from a lit exchange and are expressed
in terms of stablecoin prices:
\begin{equation}
	\frac{\bidbase + \askbase}{\bidquote + \askquote}
\end{equation}

When using either an on-chain price oracle or an exchange as a price oracle,
it is possible to perform a time or volume weighted average of prices so as
to avoid extreme price variations and to mitigate the risk and effects of any
market manipulation attemps.

% ===============================================================================

\subsection{Order crossing}
At regular time intervals, order submissions are cut off and reference prices
are determined. An element of randomness is used to determine order cutoff
times and reference price cutoff times in order to
mitigate manipulative behaviors.

An order is eligible for matching if its timestamp is within the cutoff window
and if the external reference price does not exceed its limit price.

Except on occasions where eligible buy/sell volume is perfectly matched, not
all orders can be fully crossed. There are also discretization effects to
consider arising from discrete order sizes and a discrete price grid
\cite{BoBoDoGo18}[\S 3.2.1].
See \cite{BeLaLiVa22} for a discussion of the trade-offs surrounding
different prioritization rule choices. See \cite{MsAts} for the
prioritization rules used by one of Morgan Stanley's equity liquidity pools.

\subsubsection{Priority rules}
DaoCross prioritizes fills according to:
\begin{itemize}
	\item Volume (higher volume comes first in priority)
	\item Price (higher limit price breaks volume ties)
	\item Timestamp (earlier timestamp breaks ties in volume and price)
\end{itemize}
If, in the unlikely scenario that all three of these parameters perfectly
agree, a certain priority is not guaranteed.

\subsubsection{Matching algorithm}
The mechanism for matching eligible buy and sell orders consists of two
priority queues, one for eligible buy orders and one for eligible sell orders.
Priority is determined according to the volume/price/timestamp priority rules
introduced above. Top-of-queue orders are compared, and the lesser of the two
volumes is fully filled. Once an order is fully filled at the established
reference price, it is removed from the priority queue. The procedure continues
until one of the two priority queues is empty.

\subsubsection{Computational complexity}
Let $n$ denote the sum (or max) of the number of eligible buy and sell orders.
Priority queue construction occurs in $O(n)$ time, each order removal costs
$O(\log n)$, and order removal occurs $O(n)$ times. So, the computational
time complexity of the task is $O(n \log n)$. Memory requirements are $O(n)$.

% ###############################################################################

\section{Formulation of the general DaoSwap problem}
The general problem of determining the token equilibrium prices and the
allocation among the swap participants can be formulated in terms of the
inequalities on quantities and prices corresponding to the orders and on the
need that the total quantity exchanged of each token be preserved across the
swaps, i.e., the quantity bought for each token is equal to the quantity sold.

% ===============================================================================

\subsection{Limit order inequalities}
Let $\{o_i\}_{i = 1}^n$ be a set of limit orders, each of the form
\[
	o_i = (a_i, q_i, \pi_i, p_i, \tau_i)
\]
where the quote token $\tau_i$ is implicit in the quantity $q_i$ and
the quote token $\tau_i$ and base token $\pi_i$ are implicit in the limit price
$p_i$ and in the action $a_i$.

For each order $o_i$, let $x_{i1}$ denote the realized amount of base token
exchanged, and let $x_{i2}$ denote the realized amount of quote token
exchanged.

\begin{rmk}
Regardless of the direction of the order action $a_i$, we require the quantities
of token exchanged to be nonnegative:
\[
    x_{i1} \geq 0, \quad x_{i2} \geq 0.
\]
While not a limit-order imposed constraint per se, it helps clarify
conventions, which is that a positive quantity of token must be exchanged
for a positive quantity of token.
%
%Further, we impose that $x_{i1} = 0$ if and
%only if $x_{i2} = 0$:
%\[
%    x_{i1} = 0 \iff x_{i2} = 0.
%\]
%This implies that, when exchange takes place, the effective price is finite and
%nonnegative, that is, there is ``no free lunch".
\end{rmk}

\begin{rmk}
With this notation, the limit order constraint on quantity may be expressed as
\[
    x_{i1} \leq q_i
\]
\end{rmk}

Additionally, for each $o_i$, let $a_{i1}$ and $a_{i2}$ belong to
$\{-1, +1\}$, with sign determined by $o_i$ in the following way: if
$a_i$ (the action) of $o_i$ is to buy, then set $a_{i1} = -1$ (the buy
order is taking liquidity of the base token), and if it is to sell, then
set $a_{i1} = 1$. We always have $a_{i2} = -a_{i1}$, and if the action
of $a_i$ is to sell rather than to buy, then its signs are reversed with
respect to the buy action.

The effective transaction price (in the event that positive tokens are
exchanged) is represented by $x_{i2} / x_{i1}$. If the action is to buy (base
token), then $p_i$ represents an upper bound, and the opposite is true if the
action is to sell.

\begin{rmk}
The limit order constraint on price may be expressed as
\[
    a_{i2} x_{i2} \leq x_{i1} p_i
\]
This holds true over the whole range of valid values for the $x_{ij}$
(that is, including equality with zero).
\end{rmk}

% ===============================================================================

% ===============================================================================

\subsection{Supply and demand match}

Let $T$ denote the intersection of the union of all base tokens and the union
of all quote tokens:
\[
	T :=
    \left( \bigcup_{i=1}^n \pi_i \right)
	\bigcap
    \left( \bigcup_{i=1}^n \tau_i \right)
\]
The set of tokens $T$ is the set of tokens eligible for swapping.

For each token $u \in T$, define the indicator function
$\mathcal{T}: T \to \{0, 1\}$ via
\[
	\mathcal{T}_u(v) =
	\begin{cases}
		1 \text{ if } v = u \\
		0 \text{ if } v \neq u
	\end{cases}
\]
Then, for each $u \in T$, the following conservation law must hold:
\[
    \sum_i \mathcal{T}_u(\pi_i) \cdot a_{i1} x_{i1}
    + \mathcal{T}_u(\tau_i) \cdot a_{i2} x_{i2}
    = 0
\]
Note that this translates into one equality per token participating in the
swap. These equalities express the notion that each filled order must have,
across the collection of orders, suitable counterparties.

% ===============================================================================

\subsection{Combining the constraints}

We now collect the constraints that must be satisfied:
\begin{equation}
	\begin{cases}
        x_{i1} \geq 0, \quad x_{i2} \geq 0 & \forall i \in \{1, \ldots, n\} \\
        % x_{i1} = 0 \iff x_{i2} = 0 & \forall i \in \{1, \ldots, n\} \\
        x_{i1} \leq q_i & \forall i \in \{1, \ldots, n\} \\
        a_{i2} x_{i2} \leq x_{i1} p_i & \forall i \in \{1, \ldots, n\} \\
        %x_{i1} x_{j2} - x_{i2} x_{j1} = 0 & \forall i, j \in \{1, \ldots, n\}: \pi_i = \pi_j \land \tau_i = \tau_j \\
        %x_{i1} x_{j1} - x_{i2} x_{j2} = 0 & \forall i, j \in \{1, \ldots, n\}: \pi_i = \tau_j \land \tau_i = \pi_j \\
        \sum_{i=1}^n \mathcal{T}_u(\pi_i) \cdot a_{i1} x_{i1}
        + \mathcal{T}_u(\tau_i) \cdot a_{i2} x_{i2}
        = 0 & \forall u \in T
	\end{cases}
\end{equation}

% ===============================================================================

\subsection{Solution of the general DaoSwap problem}

Given the constraints, our goal is to find quantities $x_{i1}, x_{i2}$ such
that exchanged quantity is maximized.

Note that this problem falls in domain of linear programming. We may arrange
the nonnegative quantities $x_{i1}, x_{i2}$, $i \in \{1, \ldots, n\}$ in a
vector $x$. Our goal is to find $x$ that maximizes
\[
    c^T x
\]
where $c^T = \begin{bmatrix}1 & \ldots & 1 \end{bmatrix}$ is of size $2n$
subject to
\[
    A x \leq b
\]
(TODO(Paul): reformulate the constraints using $A$ and $b$)
and
\[
    x \geq 0.
\]
The problem is polynomial-time in the ideal setting of infinite divisibility.
Discretization (and renormalization) move the problem to the domain of
integer programming, which is NP-hard.

In our setting, where the typical exchanged value is much larger than the
discretization grid, obtaining a linear programming solution and
discretizing after-the-fact may be acceptable in practice.



% ===============================================================================

\subsection{Additional constraint 1: Unique clearing price per token pair}

In the DaoCross case, a single exchange rate per exchanged token pair
prevails. One may make a case that a desirable property for a swap is that,
for participants trading the same token pair, no participant gets a better or
worse price than any other.

Suppse the effective exchange rate for an order involving tokens $\nu \in T$
and $\eta \in T$ is the same for all filled orders. In other words, the clearing
exchange rate is unique for all participants.
Then, for all $i, j$ whose respective orders $o_i, o_j$ both involve tokens
$\nu$, $\eta$, we have
\[
    x_{i1} x_{j2} - x_{i2} x_{j1} = 0
\]
when the tokens $\nu$ and $\eta$ play identical base/quote roles in $o_i$,
$o_j$, and
\[
    x_{i1} x_{j1} - x_{i2} x_{j2} = 0
\]
if they play opposite roles.

We rewrite this as
\begin{align*}
    x_{i1} x_{j2} - x_{i2} x_{j1} = 0 & & \forall i, j \in \{1, \ldots, n\}: \pi_i = \pi_j \land \tau_i = \tau_j \\
    x_{i1} x_{j1} - x_{i2} x_{j2} = 0 & & \forall i, j \in \{1, \ldots, n\}: \pi_i = \tau_j \land \tau_i = \pi_j
\end{align*}

The unique clearing price constraint prevents the problem from falling in the
domain of linear programming.

Note that limit order constraints as formulated in the DaoCross case also hint
at this nonlinearity, as they involve conjunctions with a global clearing
price (which DaoSwap must determine).

TODO(gp): explain better, cite...
The general solution of the DaoSwap problem must satisfy a set of non-linear,
combinatorial constraints, whose solution is NP-complete.

It can still be solved in an effective way with solvers like...

% ===============================================================================

\subsection{Additional constraint 2: No arbitrage}

Beyond the constraint of having a unique clearing price (if the clearing price
is defined), it is desirable to structure trade so that arbitrage opportunities
in the realized exchange were not present.

Let $\Gamma$ denote the directed graph whose vertices $V$ are given by the set
of elements $T$, and whose directed edges $E$ are given by
\[
    E = \bigcup_{\substack{i \in \{1, \ldots, n\} \\ \pi_i, \tau_i \in T}}
    \left( (\pi_i, \tau_i) \cup (\tau_i, \pi_i) \right)
\]
Each edge $(b, a) \in E$ seek to give the weight $w_{a, b}$. Note that
$w_{a, b}$ and $w_{b, a}$ are distinct.

Let $C$ denote the collection of all simple cycles of $\Gamma$.
For an simple cycle $\gamma \in C$, consecutively enumerate the vertices
(with starting place arbitrary) by $c_j$ for $j = 1, \ldots, |\gamma|$.
Let $c_0 := c_{|\gamma|}$ for notational convenience. Then
the \emph{no arbitrage} constraints states that, for each $\gamma \in C$, we have
\[
    \sum_{j = 0}^{|\gamma|} w_{j, j + 1} = 0
\]
In the case of a two-token cycle, this constraint reduces to ensuring that
there is a unique exchange rate between the pair of tokens (changing the roles
of the base and quote tokens in limit orders inverts the exchange rate).
For an efficient algorithm for finding all elementary cycles, see \cite{Jo75}
as well as \cite{MaDe76} for a comparison.

% ===============================================================================

\subsection{DaoCross problem as simplified DaoSwap problem}
In the DaoCross set-up, the price of the tokens involved in the swap is
determined by an external oracle.
This allows simplifying the general problem by applying the equivalence
principle between orders and removing the non-linearity from the set of
inequalities above.

In the case of each order $o_i$, the boolean condition $c_i$ (which compares
the actual price $p^*$ and the limit order price $p(o_i)$) is either verified
or not (the solution does not need to consider both branches):
\begin{equation}
	\begin{cases}
		0 \le q_{base}^*(o_i) \le q_{base}^* & \text{ if } c_i       \\
		q_{base}^*(o_i) = 0                & \text{ if } \lnot c_i \\
	\end{cases}.
\end{equation}
TODO: Comment on quote token amount exchanged. Verify supply meets demand.

The problem then becomes a set of linear inequalities (in fact, if $\lnot c_i$
prevails, then order $o_i$ is effectively removed from problem) that can be
solved with various efficient methods (e.g., simplex-method).

TODO(gp): check

% ###############################################################################

\section{Fees and tokenomics}
DELOC charges a fee on each transaction based on the exchanged tokens.
E.g., a transaction requiring

% ###############################################################################

\section{Implementation details}

% ===============================================================================

\subsection{DaoCross and DaoSwap Architecture}



\subsection{Off-chain computation}
Currently DELOC offloads some computations to external oracles, due to current
computational limitation of blockchains. We do not believe that this is detrimental
to the security and decentralization level of DELOC as long as these computations
are provably correct.

For instance, although the solution of the DaoSwap problem is NP-hard, verifying
that one solution is correct only requires time linear in the number of
constraints.

For this reason, DaoSwap stores on-chain the result of the DaoSwap optimization
in order to allow independent verification that the off-chain system is not
malicious or compromised.

\subsection{Ensuring a timely solution}
DaoSwap avoids the case where solving the optimization problem becomes
intractable by performing a swap when a maximum number of orders and/or
currency pairs is reached.

\subsection{Performing the swap}

% ###############################################################################

\section{Alternative solutions for token swapping}

TODO(gp): Add colors: in green what's good, yellow, red

TODO(gp): Reorg, maybe from best to worst? DaoSwap, LOB, DaoCross, AMM

% TODO(gp): Consider using one of the fancy table, e.g., tabularray
\begin{samepage}
	\begin{table}[h!]
		\centering
		\begin{tabular}{lllll}
			                            & \emph{LOB} & \emph{AMM} & \emph{DaoSwap} & \emph{DaoCross} \\
			\hline
			Custodial risk              & Yes        & No         & No             & No              \\
			Transparency                & Low        & High       & High           & Medium          \\
			Censorship resistance       & No         & Yes        & Yes            & No              \\
			Intrinsic efficiency        & Medium     & Low        & High           & Medium          \\
			Support for limit orders    & Yes        & No         & Yes            & Yes             \\
			Risk of predatory behaviors & High       & High       & Low            & Medium          \\
			Impermanent loss            & No         & Yes        & No             & No              \\
			Multi-coin exchange support & Low        & Medium     & High           & High            \\
			\hline
		\end{tabular}
	\end{table}
\end{samepage}

% *******************************************************************************
\subsection{DaoSwap}

\begin{itemize}
	\item Custodial risk: Low
	      \begin{itemize}
		      \item Private keys and funds are always under the control of the users
	      \end{itemize}
	\item Transparency: High
	      \begin{itemize}
              \item Work is done on-chain and the off-chain computations are
                independently verifiable
	      \end{itemize}
	\item Censoriship resistance: Yes
	      \begin{itemize}
		      \item The application runs on-chain and even if the web front-end is attacked
		            or disabled. The off-chain computation can be made robust using similar
		            approaches to distributed oracles, or even ported on-chain if gas prices
		            are not an issue (e.g., using Layer2 solutions)
	      \end{itemize}
	\item Intrinsic efficiency: High
	      \begin{itemize}
              \item DaoSwap relies on periodic auctions and it has been argued
                in previous literature that for small enough intervals (e.g.,
                seconds) periodic auctions result in the same quality of
                continuous matching without allowing predatory behaviors
		      \item TODO(gp): Add a reference to Cramton paper
	      \end{itemize}
	\item Support for limit order: Yes
	      \begin{itemize}
		      \item Limit orders are supported natively
	      \end{itemize}
	\item Risk of predatory behaviors: Low
	      \begin{itemize}
		      \item TODO(gp):
	      \end{itemize}
	\item Impermanent loss: No
	      \begin{itemize}
		      \item TODO(gp):
	      \end{itemize}
	\item Support for multi-coin swap: Yes
	      \begin{itemize}
              \item TODO(gp): DaoSwap/Cross pools all the liquidity in a single
                optimization problem
	      \end{itemize}
\end{itemize}

% *******************************************************************************
\subsection{DaoCross}

\begin{itemize}
	\item Custodial risk: Low
	      \begin{itemize}
		      \item Same as DaoSwap
	      \end{itemize}
	\item Transparency: Medium
	      \begin{itemize}
		      \item DaoCross relies on price discovery carried out on a lit exchange
	      \end{itemize}
	\item Censoriship resistance: No
	      \begin{itemize}
              \item The robustness of DaoCross is the same as the lit exchange
                that it relies on. See comments on the corresponding topic for
                LOB
	      \end{itemize}
	\item Intrinsic efficiency: Medium
	      \begin{itemize}
              \item The efficiency in matching trades is the same as the
                limit-order book. See comments on the corresponding topic for
                LOB
	      \end{itemize}
	\item Support for limit order: Yes
	      \begin{itemize}
		      \item Same as DaoSwap
	      \end{itemize}
	\item Risk of predatory behaviors:
	      \begin{itemize}
		      \item TODO(gp)
	      \end{itemize}
	\item Impermanent loss: No
	      \begin{itemize}
		      \item TODO(gp)
	      \end{itemize}
	\item Support for multi-coin swap:
	      \begin{itemize}
              \item TODO(gp): DaoSwap/Cross pools all the liquidity in a single
                optimization problem
	      \end{itemize}
\end{itemize}

% *******************************************************************************
\subsection{Limit Order Book (LOB)}

\begin{itemize}
	\item Custodial risk: High
	      \begin{itemize}
		      \item Exchanges own user private keys and funds, creating issues with fraud (e.g., FTX) and hacking (Mt. Gox)
	      \end{itemize}
	\item Transparency: Low
	      \begin{itemize}
              \item Many centralized exchanges can easily exaggerate volume
                (e.g., through wash trading) to attract ICOs and liquidity
		      \item There is no guarantee on the
	      \end{itemize}
	\item Censorship resistance: No
	      \begin{itemize}
              \item Users need to link their bank account to CEX and transfer
                funds, which can take 2-3 days and be subjected to
                Know-Your-Customer (KYC) and AML (Anti money laundering
                policies)
              \item Governments can censor, interfere with, or even severe the
                connection between traditional and decentralized finance (see
                recent interventions of SEC against Coinbase and Binance, and
                FDIC rescue of Silvergate Bank, Signature Bank)
	      \end{itemize}
	\item Intrinsic efficiency: Medium
	      \begin{itemize}
              \item Limit order books are considered efficient in matching
                continuously trades, although recently researchers have
                criticized their time-continuous-time nature as source of
                latency arbitrage (TODO(gp): Add ref to Compton, IEX, Flash
                Boys)
              \item The same token can be traded on different exchanges causing
                liquidity and price discovery to be fragmented with detriment
                on market quality
	      \end{itemize}
	\item Support for limit order: Yes
	      \begin{itemize}
		      \item Limit order books naturally support limit orders from users
		      \item Risk of predatory behaviors: High
		      \item LOB operators often welcome and incentivize high-frequency traders as way
		            to increase trading revenues and liquidity, at expense of predatory tactics
		            (e.g., front-running, latency arbitrage, spoofing, sniping) which arm
	      \end{itemize}
	\item Impermanent loss: No
	      \begin{itemize}
		      \item LOB
	      \end{itemize}
	\item Support for multi-coin swap: No
\end{itemize}

% *******************************************************************************
\subsection{Automatic Market Makers (AMM)}

\begin{itemize}
	\item Custodial risk: Low
	      \begin{itemize}
		      \item Private keys and the
	      \end{itemize}
	\item Transparency:
	\item Censoriship risk:
	\item Intrinsic efficiency: Low
	      \begin{itemize}
              \item Arbitrageurs are needed to keep liquidity pool and prices
                in alignment with other centers for price discovery
	      \end{itemize}
	\item Support for limit order: No
	      \begin{itemize}
		      \item Uniswap V3 doesn't support limit orders directly but only in terms of
	      \end{itemize}
	\item Risk of predatory behaviors: High
	      \begin{itemize}
		      \item TODO(gp):
	      \end{itemize}
	\item Impermanent loss: Yes
	      \begin{itemize}
              \item Uniswap is affected by impermanent loss because liquidity
                providers and liquidity takers are in general different users
                operating at different time scales
	      \end{itemize}
	\item Support for multi-coin swap: Medium
	      \begin{itemize}
		      \item Uniswap requires multiple swaps and fees for arbitrary tokens
	      \end{itemize}
\end{itemize}


% ###############################################################################

\section{Comments on AMMs}

Some of the benefits of AMMs are:
\begin{itemize}
	\item conceptual simplicity
	\item low computational requirements
	\item ability to provide liquidity even for illiquid markets
	\item ability to function without a reference price
\end{itemize}

Some of the drawbacks of AMMs are:
\begin{itemize}
	\item force liquidity providers to trade at worse-than-market prices
	\item a rarely used building block used mainly in prediction markets rather than
	      in mainstream finance. In fact only recently papers have started analyzing
	      the financial return / risk profile of AMMs \cite{MiMoRoZh22}
	\item price needs to be corrected by an arbitrageur, impacting the quality of
	      the provided liquidity
    \item artificially separates liquidity providers from traders, preventing traders
          from providing liquidity to each other in the way well-functioning
          markets do
\end{itemize}

Some critiques and counterpoints can be made to the benefits of AMMs listed above.
\begin{itemize}
	\item conceptual simplicity. Although this is a favorable point for users
	      without experience in finance, the evolution of finance practices favors the
	      use of limit orders, as explained in the introduction.
	\item low computational requirements. This is not necessarily a strict
	      requirement any more due to progress in off-chain computation and
          improved scalability in blockchain technology (such as layer 2
          blockchains). We do not believe that an inefficient solution (from
          the point of view of exchanged value) should be preferred only
          because of implementation simplicity.
	\item ability to function even for illiquid markets.
	      Although this is a valid advantage for many markets (e.g., prediction markets),
	      the vast majority of crypto coins are extremely liquid and do not require
	      trading off trading quality.
	\item ability to function without a reference price.
          This feature was certainly appealing to initial researchers (e.g., \cite{Bu17})
	      in search of a fully-decentralized solution to the problem
	      of token exchange. In reality, the fact that price discovery unquestionably
	      happens on current CEXs turned AMMs into arbitrage generation machines
	      (AGMs), where 80\% of the trading volume is due to arbitrageurs keeping AMM
	      prices in sync with the predominant price TODO(gp): Add reference
\end{itemize}

One of the problems with current state of DeFi is that users look for
applications to use their crypto coins, waiting for mainsteam adoption of
crypto in every day payment.
Also a current trend is for crypto holders to find yield to benefit from
holding crypto. AMMs satisfied the need for ways to extract yields from holding
coins.

We don't think AMM will completely replaced but new mechanisms are needed LPs
need to be compensated with revenues to offset the cost of their adverse
selection (quantified and characterized by the LVR paper)

\section{CPMM as an LOB}

TODO(Paul): Replace with a reference to the derivation in \cite{Yo20}, and
focus on the difference in stylized facts between the LOB a CPMM implies
versus the LOBs seen in LOB-based markets. Note that some conventions here
vs the paper are slightly different.

Here we reinterpret a constant product market maker as a limit order order book
that a market participant may transact against (e.g., hit a bid or lift an
offer). In a traditional limit order book, one expects frequent limit order
cancellation and submission in response to market activity.

Let $\tA$ and $\tB$ be two tokens in a CPMM and suppose that the available
supply of $\tA$ and $\tB$ in the pool at time $t_0$ are, respectively,
$a_0$ and $b_0$. We set $k = a_0 b_0$. Though neither token $\tA$ nor token
$\tB$ enjoys a distinguished role, for the purposes of this discussion we
designate token $\tA$ the base token and token $\tB$ the quote token
(due to the symmetry, the subsequent analysis continues to hold with the roles
reversed). From this perspective, we think of placing orders to either buy or
sell token $\tA$, with price quoted in token $\tB$ per unit of $\tA$.

\subsection{Buy order}

Suppose we wish to place an order to buy $\delta a$ of token $\tA$. This
corresponds to removing an amount of $\delta a$ from the pool, which must
be compensated for by adding (paying) some amount $\delta b$ to the pool.
The constant product constraint ensures
\[
	a_0 b_0 = (a_0 - \delta a) (b_0 + \delta b)
\]
which upon rearrangement leads to
\[
	\delta b = b_0 \frac{\delta a}{a_0 - \delta a}
\]
The effective cost of buying $\delta a$ units of $\tA$ is given by
\[
	\frac{\delta b}{\delta a} = \frac{b_0}{a_0 - \delta a}
\]
If we make the substitution $b_0 = k / a_0$, this becomes
\[
	\frac{\delta b}{\delta a} = \frac{k}{a_0(a_0 - \delta a)}
\]
In the idealized setting of infinite divisibility of tokens, the
instantaneous price of buying $\tA$ at point $(a_0, b_0)$ is
\[
	\frac{b_0}{a_0} = \frac{k}{a_0^2}
\]
For convenience, define $p_0$ as this price:
\[
    p_0 := \frac{b_0}{a_0} = \frac{k}{a_0^2}
\]
Note that as more of token $\tA$ is purchased (more $\tA$ is withdrawn from
the pool), the price of token $\tA$ in the pool (in terms of the quote token)
goes up, as expected.

\subsection{Supply liquidity}

TODO(Paul): Here we use $p$ as effective price of transacting quantity $q$.
Using cumulative quantity up to a given (max) price $p$ (as in \cite{Yo20}) is
more natural for some analyses.

Rephrasing, suppose $q$ denotes quantity of token $\tA$ that we wish to
purchase. If the current state of the market is $a_0, b_0$, then the price $p$
of purchasing quantity $q$ is
\[
	p(a_0, b_0, q) = \frac{k}{a_0 (a_0 - q)}
\]
For brevity, we drop the functional dependence. Reexpressing in
terms of quantity, we have
\[
    q = a_0 - \frac{k}{a_0 p} = a_0 \left(1 - \frac{p_0}{p}\right)
\]
which expresses cumulative quantity $q$ available at average price $p$.

\subsubsection{Slippage and post-trade state}

Measure slippage as the relative difference between effective buy price and
market price $p_0$. We calculate
\[
    \frac{p - p_0}{p_0} = \frac{a_0}{a_0 - q} - 1 = \frac{q}{a_0 - q}
\]
Note that this only depends upon amount of $\tA$ in the pool and the purchase
quantity. TODO(Paul): find and cite the earliest references for this.
Note that this is approximately linear in $q$ for large $a_0$ (in other
words, market impact is approximately linear when there is sufficient
depth). Note that this is independent of volatility and turnover.

When a buy transaction takes places, the market price changes from
\[
	p_0 = \frac{b_0}{a_0}
\]
to
\[
	p_1 = \frac{b_0 + \delta b}{a_0 - \delta a}.
\]
Note that this is greater than the effective purchase price of
\[
	\frac{b_0}{a_0 - \delta a},
\]
which is analogous to what happens when multiple levels of a limit order book
are cleared (the new best offer is greater than the effective price of the buy
order that cleared multiple levels of offers).

Following a transaction, marginal quantity readjusts given the new position
$(a_1, b_1)$ of the book.

% ###############################################################################

\bibliography{Daobib}
\bibliographystyle{amsplain}

\begin{thebibliography}{10}

	\bib{Ad18}{article}{
		author={Adams, Hayden},
		title={Uniswap Whitepaper},
		date={2018},
		eprint={https://hackmd.io/s/HJ9jLsfTz},
	}

	\bib{AdZiRo20}{article}{
		author={Adams, Hayden},
		author={Zinsmeister, Noah},
		author={Robinson, Dan},
		title={Uniswap v2 Core},
		date={March 2020},
		eprint={https://uniswap.org/whitepaper.pdf},
	}

	\bib{AdZiSaKeRo21}{article}{
		author={Adams, Hayden},
		author={Zinsmeister, Noah},
		author={Salem, Moody},
		author={Keefer, River},
		author={Robinson, Dan},
		title={Uniswap v3 Core},
		date={March 2021},
		eprint={https://uniswap.org/whitepaper-v3.pdf},
	}

	\bib{BeLaLiVa22}{article}{
		author={Bernales, Alejandro},
		author={Ladley, Daniel},
		author={Litos, Evangelos},
		author={Valenzuela, Marcela},
		title={Alternative Execution Priority Rules in Dark Pools},
		date={July 22, 2022},
		eprint={https://papers.ssrn.com/sol3/papers.cfm?abstract_id=4169352},
	}

	\bib{BoBoDoGo18}{book}{
		author={Bouchaud, Jean-Philippe},
		author={Bonart, Julius},
		author={Donier, Jonathan},
		author={Gould, Martin},
		title={Trades, Quotes and Prices: Financial Markets Under the Microscope},
		date={2018},
		publisher={Cambridge University Press},
		doi={https://doi.org/10.1017/9781316659335},
	}

    \bib{Bu17}{article}{
        author={Buterin, Vitalik},
        title={Let's run on-chain decentralized exchanges the way we run prediction markets},
        date={2017},
        eprint={https://www.reddit.com/r/ethereum/comments/55m04x/lets_run_onchain_decentralized_exchanges_the_way}
    }

	\bib{Cme23}{article}{
		author={CME Group},
		title={2023 FX Product Guide},
		edition={22nd},
		eprint={https://www.cmegroup.com/trading/fx/files/fx-product-guide-2023-us.pdf},
	}

	\bib{CmeFx}{article}{
		author={CME Group},
		title={Euro FX Futures - Contract Specs},
		eprint={https://www.cmegroup.com/markets/fx/g10/euro-fx.contractSpecs.html},
	}

	\bib{Ha03}{article}{
		author={Hanson, Robin},
		title={Combinatorial Information Market Design},
		journal={Information Systems Frontiers},
		date={2003},
		volume={5},
		pages={107-119},
		eprint={https://doi.org/10.1023/A:1022058209073},
	}

    \bib{Jo75}{article}{
        author={Johnson, Donald B.},
        title={Finding all the Elementary Circuits of a Directed Graph},
        journal={SIAM Journal on Computing},
        date={1975},
        volume={4},
        number={1},
        pages={77-84},
    }

    \bib{MaDe76}{article}{
        author={Mateti, Prabhaker},
        author={Narsingh, Deo},
        title={On Algorithms for Enumerating All Circuits of a Graph},
        journal={SIAM Journal on Computing},
        date={1976},
        volume={5},
        number={1},
        pages={90-99},
        eprint={https://doi.org/10.1137/0205007}
    }

    \bib{MiMoRo23}{article}{
		author={Milionis, Jason},
		author={Moallemi, Ciamac C.},
		author={Roughgarden, Tim},
        title={Complexity-Approximation Trade-offs in Exchange Mechanisms: AMMs vs. LOBs},
        date={2023},
        eprint={https://arxiv.org/abs/2302.11652},
    }

	\bib{MiMoRoZh22}{article}{
		author={Milionis, Jason},
		author={Moallemi, Ciamac C.},
		author={Roughgarden, Tim},
		author={Zhang, Anthony Lee},
		title={Automated Market Making and Loss-Versus-Rebalancing},
		date={2022},
		eprint={https://doi.org/10.48550/arXiv.2208.06046},
	}

	\bib{Ms}{article}{
		author={Morgan Stanley},
		title={Morgan Stanley Dark Pools},
		eprint={https://www.morganstanley.com/disclosures/morgan-stanley-dark-pools},
	}

	\bib{MsAts}{article}{
		author={Morgan Stanley},
		title={MS Pool ATS-N Filings},
		eprint={https://www.sec.gov/cgi-bin/browse-edgar?action=getcompany&filenum=013-00117}
	}

	\bib{Wa}{article}{
		title={Walrasian auction},
		eprint={https://en.wikipedia.org/wiki/Walrasian_auction},
	}

	\bib{XiStWh05}{article}{
		author={Xia, Mu},
		author={Stallaert, Jan},
		author={Whinston, Andrew B.},
		title={Solving the combinatorial double auction problem},
		journal={Journal of Operation Research},
		date={2005},
		pages={239-251},
		volume={164},
		eprint={https://www.sciencedirect.com/science/article/abs/pii/S0377221703008981},
	}

    \bib{Yo20}{article}{
        author={Young, Jamie E.},
        title={On Equivalence of Automated Market Maker and Limit Order Book Systems},
        date={October 2020},
        eprint={https://professorjey.com/assets/papers/AMM_Order_Book_Equivalence_DRAFT_2020_10_16.pdf},
    }


\end{thebibliography}

\end{document}
